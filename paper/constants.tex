\newcommand{\customer}{customer}
\renewcommand{\sp}{SP}

\newcommand{\normal}{n}
\newcommand{\dispute}{d}
\newcommand{\abnormal}{\overline{n}}
\newcommand{\abdispute}{\overline{d}}

\newcommand{\PoD}{$\mathrm{PoD}$}
\newcommand{\PoP}{$\mathrm{PoP}$}
\newcommand{\cid}{$\mathrm{cid}$}

\newcommand{\floor}[1]{\left\lfloor #1 \right\rfloor}
\newcommand{\ceil}[1]{\left\lceil #1 \right\rceil}

\newcommand{\plus}{+}
\newcommand{\minus}{-}
\newcommand\neutral[1][.75]{\mathbin{\ThisStyle{\vcenter{\hbox{%
  \scalebox{#1}{$\SavedStyle\bullet$}}}}}%
}

\def\SINGLECOLUMN{SINGLE}
\def\DOUBLECOLUMN{DOUBLE}
\def\JOURNALIEEE{IEEE}
\def\JOURNALELS{ELS}

\newcommand\TITLE{Anonymous provision of privacy-sensitive services using distributed networks}
% \newcommand\TITLE{Privacy-preserving physical delivery service provisioning using distributed networks}
\newcommand\STANLONG{Stanis\l{}aw Bara{\'n}ski}
\newcommand\JULLONG{Julian Szyma{\'n}ski}
\newcommand\STAN{S. Bara{\'n}ski}
\newcommand\JUL{J. Szyma{\'n}ski}

\newcommand\ABSTRACTPOL{%
Swiadczenie wrazliwych uslug takich jak testy genetyczne, testy na rodzicielstwo, testy na choroby weneryczne, porady prawne, czy oszacowanie zdolnosci kredytowej, czesto wymaga dostarczenia danych zawierających dane osobiste.

Przetwarzanie danych osobistych, szczególnie przy śwaidczeniu usług wykorzystujacych wrazliwe dane (np. health records, biological materials, or legal documents), naraza zarówno uzytkownikow jak i dostawcow uslug (SP) na ryzyko.

SPs narazeni sa na zlamanie GDPR, CPAA, i innych prawnych regulacji, natomiast Klienci narazeni sa na utrate prywatnosci.

Obserwujemy ze dane osobiste często wykorzystywane sa jedynie do celów logistycznych jak płatnosci czy komunikacja, i moglyby by być swiadczone anonimowo jeśli takie metody były by dostepne.

W tej pracy proponujemy rozwiazanie pozwalajace na swiadczenie uslug bez zbierania danych osobistych. Wykorzystujemy anonimowe metody platnosci jak gotówka oraz anonimowe kryptowaluty, technologie blockchain do osiągniecia sprawiedliwosci, oraz rozproszone content-addressable storage networks do dostarczania wynikow.

W porownaniu do innych rozwiazan, nasz protokol osiaga anonimowosc przy slabszych zalozeniach, wspiera fizyczne materialy, rozwiazywanie konfliktow, oraz w optymistycznym scenariuszu nie wymaga, aby uzytkownik wchodzil w interakcje z blockchainem, wykazujac lepsza praktycznosc.

W pracy analizujemy sprawiedliwosc naszego protokolu, oraz implementujemy go przy uzyciu technologi Ethereum jako message board, Monero jako anonimowy sposób płatności, and Powergate (IPFS i Filecoin) jako content-addressable network.

Rozpatrujemy dalsze rozszerzenia do anonimowych dostarczycieli, self-sovereign identifiers, oraz zdecentralizowanego rozwiazywania konfliktow.
}

\newcommand\ABSTRACT{
Lawyers, laboratories, auditors or banks, in order to provide services, require data often containing sensitive personal information.

Examples of sensitive services are genetic tests, parenthood tests, STD tests, credit score computation or legal advice. 

The processing of personal data, especially when providing services involving sensitive data like health records, biological materials, or legal documents, exposes both users and service providers (SP) to privacy risks.

SPs are at risk of violating GDPR, CPAA, and other legal regulations, while customers are at risk of losing their privacy.

We observe that personal data is often only used for logistical purposes, such as payments or communication, and could be provided anonymously if such methods were available.

In this work, we propose a solution that allows the provision of services without the collection of personal data. We use anonymous payment methods such as cash and anonymous cryptocurrencies, blockchain technology to achieve fairness, and distributed content-addressable storage networks to deliver results.

Compared to other solutions, our protocol achieves anonymity with weaker assumptions, supports physical materials and conflict resolution, and in conflict-free transactions does not require the customer to interact with the blockchain, demonstrating better practicality.

In this work, we analyse the fairness of our protocol and implement it using Ethereum technology as a message board, Monero as an anonymous payment method, and Powergate (IPFS and Filecoin) as a decentralised storage network.
}


\newcommand\ACKNOWLEDGEMENTS{
The work has been supported partially by the founds of Department of Computer Architecture Faculty of Electronics, Telecommunications and Informatics, Gdańsk University of Technology.
}

\newcommand\LASTVISITED{(last visited Jan. 31, 2023)}