\newcommand{\customer}{customer}
\renewcommand{\sp}{SP}

\newcommand{\normal}{n}
\newcommand{\dispute}{d}
\newcommand{\abnormal}{\overline{n}}
\newcommand{\abdispute}{\overline{d}}

\newcommand{\PoD}{$\mathrm{PoD}$}
\newcommand{\PoP}{$\mathrm{PoP}$}
\newcommand{\cid}{$\mathrm{cid}$}

\newcommand{\floor}[1]{\left\lfloor #1 \right\rfloor}
\newcommand{\ceil}[1]{\left\lceil #1 \right\rceil}

\newcommand{\plus}{+}
\newcommand{\minus}{-}
\newcommand\neutral[1][.75]{\mathbin{\ThisStyle{\vcenter{\hbox{%
  \scalebox{#1}{$\SavedStyle\bullet$}}}}}%
}

\newcommand\TITLE{Anonymous provision of privacy-sensitive services using distributed networks}
% \newcommand\TITLE{Privacy-preserving physical delivery service provisioning using distributed networks}
\newcommand\STANLONG{Stanis\l{}aw Bara{\'n}ski}
\newcommand\JULLONG{Julian Szyma{\'n}ski}
\newcommand\STAN{S. Bara{\'n}ski}
\newcommand\JUL{J. Szyma{\'n}ski}

\newcommand\ABSTRACT{%
Service providers like lawyers, laboratories, auditors, or banks, to provide services, require customers to submit data often associated with their personal information.
This information puts suppliers at risk of breaking GDPR, CPAA, or other privacy regulations, and puts customers at risk of privacy violations.

However, most service providers use personal information merely for logistic operations like payments and communication and could provide services anonymously if other means of logistics exist.
We propose an anonymous protocol for coordinating service provision using blockchain for both anonymous payments and proofs of existence and content-addressable network for results delivery.
% Less important
Service providers may use our protocol to provide services without collecting personal information. Enabling provision of services that were previously limited by a high level of trust.
% Even less important
Possible applications for this protocol include anonymous genetic tests, tests for paternity, venereal diseases, HIV, drugs, and steroids, or anonymous legal advice.

Compared to other works, our protocol achieves stronger anonimity by accepting payments in either cash or privacy-preserving blockchains. It supports physical materials, dispute resolution, and does not require customers to submit any transaction to the message board, indicating higher practicality.

We provide the fairness analysis and an implementation of the protocol using Ethereum as a message board, Monero as a privacy-preserving blockchain, and Powergate (IPFS and Filecoin) as a content-addressable network.
We discuss further extensions to an anonymous delivery system, self-sovereign identifiers, and a decentralized conflict resolution system.
%JS tu należałoby dodac kilka słów na temat rozwiazania i jak zostało ono zwalidowane czemu jest lepsze od innych
}

\newcommand\ACKNOWLEDGEMENTS{
The work has been supported partially by the founds of Department of Computer Architecture Faculty of Electronics, Telecommunications and Informatics, Gdańsk University of Technology.
}

\newcommand\LASTVISITED{(last visited Jan. 31, 2023)}