\documentclass[pdftex,twocolumn,epjc3]{svjour3} 
\RequirePackage[T1]{fontenc}
\smartqed  % flush right qed marks, e.g. at end of proof
\RequirePackage{graphicx}
\RequirePackage{mathptmx}      % use Times fonts if available on your TeX system
\RequirePackage{flushend}
\RequirePackage[numbers,sort&compress]{natbib}
\RequirePackage[colorlinks,citecolor=blue,urlcolor=blue,linkcolor=blue]{hyperref}

\journalname{Eur. Phys. J. C}


\usepackage{amsmath,amssymb,amsfonts}
\usepackage{algorithmic}
\usepackage{graphicx}
\usepackage{textcomp}
\usepackage{float}
\usepackage{etoolbox}

\usepackage{url}
\usepackage{listings}
\usepackage{pdfpages}
\usepackage[english]{babel}
\usepackage{xcolor,colortbl}
\usepackage{scalerel}
\newcommand{\customer}{customer}
\renewcommand{\sp}{SP}

\newcommand{\normal}{n}
\newcommand{\dispute}{d}
\newcommand{\abnormal}{\overline{n}}
\newcommand{\abdispute}{\overline{d}}

\newcommand{\PoD}{$\mathrm{PoD}$}
\newcommand{\PoP}{$\mathrm{PoP}$}
\newcommand{\cid}{$\mathrm{cid}$}

\newcommand{\floor}[1]{\left\lfloor #1 \right\rfloor}
\newcommand{\ceil}[1]{\left\lceil #1 \right\rceil}

\newcommand{\plus}{+}
\newcommand{\minus}{-}
\newcommand\neutral[1][.75]{\mathbin{\ThisStyle{\vcenter{\hbox{%
  \scalebox{#1}{$\SavedStyle\bullet$}}}}}%
}

\newcommand\TITLE{Anonymous provision of privacy-sensitive services using distributed networks}
% \newcommand\TITLE{Privacy-preserving physical delivery service provisioning using distributed networks}
\newcommand\STANLONG{Stanis\l{}aw Bara{\'n}ski}
\newcommand\JULLONG{Julian Szyma{\'n}ski}
\newcommand\STAN{S. Bara{\'n}ski}
\newcommand\JUL{J. Szyma{\'n}ski}

\newcommand\ABSTRACT{%
Service providers like lawyers, laboratories, auditors, or banks, to provide services, require customers to submit data often associated with their personal information.
This information puts suppliers at risk of breaking GDPR, CPAA, or other privacy regulations, and puts customers at risk of privacy violations.

However, most service providers use personal information merely for logistic operations like payments and communication and could provide services anonymously if other means of logistics exist.
We propose an anonymous protocol for coordinating service provision using blockchain for both anonymous payments and proofs of existence and content-addressable network for results delivery.
% Less important
Service providers may use our protocol to provide services without collecting personal information. Enabling provision of services that were previously limited by a high level of trust.
% Even less important
Possible applications for this protocol include anonymous genetic tests, tests for paternity, venereal diseases, HIV, drugs, and steroids, or anonymous legal advice.

Compared to other works, our protocol achieves stronger anonimity by accepting payments in either cash or privacy-preserving blockchains. It supports physical materials, dispute resolution, and does not require customers to submit any transaction to the message board, indicating higher practicality.

We provide the fairness analysis and an implementation of the protocol using Ethereum as a message board, Monero as a privacy-preserving blockchain, and Powergate (IPFS and Filecoin) as a content-addressable network.
We discuss further extensions to an anonymous delivery system, self-sovereign identifiers, and a decentralized conflict resolution system.
%JS tu należałoby dodac kilka słów na temat rozwiazania i jak zostało ono zwalidowane czemu jest lepsze od innych
}

\newcommand\ACKNOWLEDGEMENTS{
The work has been supported partially by the founds of Department of Computer Architecture Faculty of Electronics, Telecommunications and Informatics, Gdańsk University of Technology.
}

\newcommand\LASTVISITED{(last visited Jan. 31, 2023)}

\begin{document}

\title{\TITLE}

\author{\STANLONG\thanksref{e1,addr1}
        \and \JULLONG{}\thanksref{e2,addr1}
        \and \HIGILONG{}\thanksref{e3,addr2}
}


\thankstext{e1}{e-mail: stanislaw.baranski@pg.edu.pl}
\thankstext{e2}{e-mail: julian.szymanski@pg.edu.pl}
\thankstext{e3}{e-mail: hmora@ua.es}

\institute{Department of Electronic, Telecommunication and Informatics, Gdansk University of Technology, Narutowicza 11/12 80-233 Gdansk Poland\label{addr1}
          \and
          Department of Computer Science Technology and Computation, University of Alicante, San Vicente del Raspeig, 03690 Alicante Spain \label{addr2}
}


\date{Received: date / Accepted: date}
% The correct dates will be entered by the editor

\maketitle

\begin{abstract}
  \ABSTRACT{}
  
  \newcommand{\sep}{ \and }
  \keywords{Anonymity\sep{} blockchain\sep{} diagnosis\sep{} e-commerce\sep{} fair-exchange\sep{} privacy\sep{} services}
\end{abstract}

\section{Introduction}
\label{sec:introduction}
Diagnostic services such as paternity tests, doping tests, or venereal disease tests are performed at diagnostic centres that require clients to provide biological materials such as blood, urine, stool, or saliva. Unfortunately, to process transactions, clients are often forced to reveal their identities. Such a combination of personal data with biological materials exposes clients to significant privacy risks.

Other examples of sensitive services are, legal services, such as legal advice that may require disclosure of the history of financial transactions and confidential legal documents.

Banking services, such as credit scoring, may require sensitive personal data, such as medical records, and history of financial transactions.

Providing personal information exposes users to privacy risks, i.e. potential loss of control over personal information \cite{smithInformationPrivacyResearch2011}.

Such personal information can be used deliberately or accidentally (e.g., by theft) 
for insider disclosure, unauthorized access, or commercial gains. For example, by reselling it to marketers, financial institutions, other businesses, government agencies, or even cybercriminals. 
In turn, it can lead to an unintentional violation of the GDPR or CCPA, profiled advertisements or even criminal activities like identity theft or illegal tracking and surveillance~\cite{smithInformationPrivacyResearch2011}.

Public individuals like influencers, politicians, or celebrities are especially vulnerable to this kind of attack as exposure of their health records, purchase habits, or legal documents can threaten their reputation, position or be used for blackmailing.

The problem amplifies when personal information is linked to health records, or legal documents, or biological materials like DNA, saliva, or blood.

In such situations, customers may be reluctant to provide their personal information to the SP and thus resign from the service. Both parties are harmed as the client does not receive a diagnosis relevant to their health and the SP loses potential clients. Both parties are affected as the client does not receive a diagnosis relevant to their health and the SP loses potential clients.

The guarantee of privacy of personal data is often based solely on trust in the SPs' staff and the security of their IT systems. However, we observe that personal data is often only used for logistical purposes, such as payments or communications, and could be provided anonymously if such methods were available.

It would be desirable for the customers to keep their identities private while still receiving the service. This could lead to reduced trust that customers have to put in SPs and less responsibility borne by SPs.

Examples of services that would benefit from the anonymity property are:
\begin{itemize}
    \item Patients willing to take a test eg.: for drugs, venereal diseases, paternity or steroids have a strong incentive to keep the whole procedure private. Merely the fact that they took the test—without exposing the result—is suggestive enough to act as a premise in case of a conflict.

Currently, they have to risk that their personal details, materials, and results are stored securely and kept private from any unauthorized actor (both curious employees and malicious attackers).

\item Individuals undertaking entrepreneurial activities may wish to obtain estimates of the risks and possible repercussions of the actions taken.

By exposing their identity they trust that the lawyer will not use that information for harmful actions.
\end{itemize}

According to Solove's Taxonomy of privacy~\cite{soloveTaxonomyPrivacy2006}, our protocol aims to prevent the following privacy risks:
\begin{itemize}
    \item Identification, linking of information to a particular individual;
    \item Secondary use, using personal information for a purpose other than the purpose it was collected; 
    \item Breach of confidentiality, breaking a promise to keep a person's information confidential;
    \item Disclosure, revealing truthful information about a person that impacts the ways others judge their character.
\end{itemize}

In this work we explore the possibilities of providing services anonymously.

We observe that the issue of anonymous service provision can be seen as a problem of fair exchange where parties exchange some goods fairly, i.e., either both parties obtain the goods, or they both obtain nothing.

In our case, the customer who wants to use the service, exchanges his materials and money with the SP for the result of the service.

Our review indicates that none of the current systems is sufficient to achieve the desired goal. We propose a anonymous protocol for provisioning serivces that require physical materials. In case of dispute, either due to exceeding deadlines or providing an incorrect result, the customer can disclose the whole interaction and prove to justice the misbehavior of the SP.

To achieve these objectives, we use:
\begin{itemize}
    \item \textbf{Blockchain} (aka message board), to achieve fairness, i.e., as a means for proving that certain actions took place at a certain time without the trusted third party (TTP).
    \item \textbf{Anonymous payment methods}, like cash or anonymous cryptocurrencies, allowing customers to pay for services anonymously.
    \item \textbf{Decentralized storage network} (e.g., IPFS) to anonymously provide results together with a provable storage network (e.g., Filecoin) to guarantee that the results are available to the customer even if the SP refuses to share the result.
    
    \item Cryptography:
    \begin{itemize}
        \item \textbf{Symmetric encryption}, to encrypt and decrypt results published on the public networks.
        \item \textbf{Diffie-Helman key exchange (DHKE)} to derive shared secrets for symmetric encryption.
        \item \textbf{Digital signatures}, to achieve authentication and non-repudiation of actions. 
    \end{itemize}
    
\end{itemize}


The contribution of our paper is:
\begin{itemize}
\item proposing a novel protocol that:
  \begin{itemize}
  \item allows \textbf{anonymous} service provision, i.e., the service provider does not need to collect any personal information from the customer.
  \item achieves \textbf{fairness} by carefully designing the protocol as an interactive non-cooperative game, in which the SP remains in an advantageous position and the customer in a disadvantageous position with an option to start a dispute in case a misbehaviour of SP.
  \item \textbf{does not rely on centralized TTP}, but uses decentralized blockchain and distributed content-addressable storage network.
  \item In conflict-free transactions does not require customers to interact with the blockchain, achieving better \textbf{user-experience and practicality}.
  \item allows the \textbf{delivery of physical materials}, which was rarely addressed by other researchers or based on strong and unpractical assumptions.
  \item guarantees the \textbf{remote availability of the results}, even in case of the SP denial of service.
  \end{itemize} 
\item Systematise often misused definitions of anonymity, pseudonymity, linkability, and traceability.
\item Proposing a framework for analyzing fairness in fair-exchange protocols.
\item Discuss the possible improvements to the protocol with the use of secure computation, self-sovereign identities, zero-knowledge proofs, and blockchain-based dispute resolution techniques.
\end{itemize} 

Some authors have proposed blockchain-based fair-exchange systems that could be adapted to service provision; however, to our best knowledge, we are the first to propose a system that satisfies all the properties stated above. Especially anonymity and physical materials delivery were rarely addressed together, and if so, the protocol was based on TTP and impractical assumptions on banking system~\cite{birjoveanuAnonymityFairexchangeEcommerce2015} or did not address the conflict between parties~\cite{altawyLelantosBlockchainBasedAnonymous2017}.


The rest of this paper is organized as follows.
In Section~\ref{sec:related-works} we review related works. 
Then, in Section~\ref{sec:building-blocks} we discuss the building blocks of a dispute resolution system, blockchain as a message board, fairness, anonymous payments, storage network, results availability, and anonymity.
Section~\ref{sec:protocol} provides detailed description of the protocol.
Section~\ref{sec:fairness-analysis} provides a fairness analysis of the proposed protocol.
In Section~\ref{sec:experiments} we present the implementation of the protocol and results of our experiments.
Section~\ref{sec:discussion} discuss possible improvements in terms of crowd-sourced dispute resolution or dispute avoidance, self-sovereign identities (SSI), anonymous delivery, and formal verification.
Finally, Section~\ref{sec:conclusion} concludes the paper.


\section{Related Works}\label{sec:related-works}
In this section, we review related works on fair exchange, anonymous, and physical delivery protocols. Then we outline the main differences between the protocols.

\subsection{Fair exchange, physical delivery}
The most common application of fair exchange protocols is e-commerce. A typical transaction involves a seller and a buyer who exchange money for a physical product. To protect themselves the seller want to receive the funds prior to sending the product, while the buyer want to receive the product prior to paying. The fairness of the protocol should guarantee that either both parties obtain the goods, or they both obtain nothing.

To our best knowledge, the first system that allowed for fair exchange of physical products was proposed by Zhang et al. in 2006~\cite{zhangPracticalFairExchangeEPayment2006}. The protocol uses buyer's bank as TTP, and assumes no coalition between parties to achieve anonymity. 

Authors of~\cite{mohammedalarajFairnessPhysicalProducts2012} proposed an protocol for fair-exchange involving physical materials delivery. The protocol assumes the existence of an offline~\footnote{\textit{Offline} (also called \textit{optimistic}) TTP compared to \textit{online} TTP is in improvement which assumes the TTP being involved only in the case of a dispute between parties~\cite{rayFairExchangeEcommerce2002}.} TTP that is involved only in the event of a dispute. The protocol works as follows: 
\begin{enumerate}
\item The buyer sends an encrypted payment to the seller along with the proof of validity.
\item The seller validates the proof, which convinces him that the decrypted payment will also be valid.
\item The seller forwards the physical product to the trusted delivery agent (DA).
\item The buyer verifies whether the product is the one he ordered.
\item The buyer creates a proof of delivery and sends it along with the decryption key to the seller.
\end{enumerate}

However, the protocol relies on strong assumptions, namely, the existence of a TTP and an DA, who does not misbehave or collude with either party. Also the fairness is obtained assuming resilient communication channels.

\subsection{Fair exchange, anonymity, dispute resolution, and physical products delivery} 
\label{anonymity-and-fair-exchange-in-e-commerce-protocol-for-physical-products-delivery}

In the previous protocol, both parties of the transaction are known to each other. However, there are situations in which the buyer would prefer not to disclose his identity to the seller. The situation becomes more challenging when the transaction involves physical products.

Authors of~\cite{birjoveanuAnonymityFairexchangeEcommerce2015} proposed a protocol involving physical products, which guarantees anonymity of both buyer and seller. They achieve it with the introduction of an online TTP that validates coins and guarantees fairness. Later, they proposed many extensions to the protocol, namely, in~\cite{birjoveanuPreservingAnonymityFair2018} they proposed an anonymous e-commerce protocol that governs complex transactions, in~\cite{birjoveanuAnonymityComplexTransactions2019} they extend the anonymity for both buyers and sellers (in our case SPs), and in~\cite{birjoveanuFairExchangeECommerce2020} they proposed a protocol that supports chained transactions (involving multiple active intermediaries). Finally in~\cite{birjoveanuTwoPartyECommerceProtocols2022} they proposed improved protocol named PPPDCP (Protocol with Physical Product Delivery Providing Customer's Privacy).

Anonymity is ensured by using anonymous communication channel over Tor~\cite{dingledineTorSecondGenerationOnion2004} and blind signatures~\cite{chaumSecurityIdentificationTransaction1985}. Fair exchange, physical delivery, and dispute resolution is achieved using delivery cabinet (DC) under the seller's control (e.g. Amazon Locker) and offline TTP that intervenes only in the case of a dispute.

The protocol works as follows: 

\begin{enumerate}
    \item The buyer buys a digital coin from his bank. Blind signature is used to preserve privacy.
    \item The buyer sends to the seller the purchase details, address of DC, customer's bank signature, and some meta data to preserve the integrity of transaction, communication is done over Tor.
    \item The seller request a redemption of the digital coin.
    \item The seller posts the product to the DC via delivery serivice.
    \item The DA collects the product from the seller's cabinet and posts it to the DC.
    \item The buyer collects the product from the DC using a password.
    \item The buyer checks if the product is the ordered one, if not, opens a dispute to TTP.
\end{enumerate}

Besides the assumptions of the existence of the TTP, the protocol also assumes:

\begin{itemize}
    \item the buyer's and seller's banks enable confidential transactions, and both share a commit buffer where the value is locked until the transaction is finished;
    \item banks maintain a global list of coins' serial numbers to prevent double-spending problems;
    \item there exists an anonymous communication channel, and the SC and the DC protected by a passwords;
    \item the DC is equipped with a video camera that records the moment when the buyer opens the package and provides a way to submit the video to TTP in case of a dispute;
\end{itemize}


\subsection{Fair exchange and blockchain}

The issue of TTPs in fair exchange protocols has been solved by the usage of decentralized networks, especially blockchain technology.

The simplest example of blockchain-based fair exchange protocol is certified electronic mail protocol in which neither the sender can deny sending the mail, nor the receiver can deny receiving it. As the service is widely used in the paper world, achieving it for e-mails has not yet met agreement across the scientific community. The main problem was the dependency on TTP, which significantly reduced the performance, security, and robustness of such protocols. The protocols that do not use TTP struggle with high computation and communication overhead~\cite{hinarejosSolutionSecureCertified2019}.

Authors of ~\cite{hinarejosSolutionSecureCertified2019} replaced the TTP with blockchain (concretely Bitcoin blockchain as a reference implementation) that acts as a secure, verifiable, and decentralized TTP.

The idea behind certified e-mails and any other fair exchange protocols is the following:
\begin{enumerate}
    \item The sender sends an encrypted and signed message to the receiver.
    \item The receiver returns a proof of delivery (a signature) of the encrypted message to the sender.
    \item The sender publishes proof of delivery and the decryption key on a blockchain (or TTP in general).
    \item The receiver decrypts the encrypted message using the published decryption key.
\end{enumerate}

The non-repudiation requirement is achieved by the receiver sending the proof of delivery prior to having access to the decrypted message and the sender publishing the decryption key together with the proof of delivery on the blockchain (or TTP) so that the receiver can deny neither receiving the message nor having access to the decryption key—because it is publicly available.

In this case, the role of blockchain is to certify the existence of the decryption key at a certain time.

\subsection{Fair-exchange, blockchain, and decentralized dispute resolution}
\label{themis-towards-decentralized-escrow-of-cryptocurrencies-without-trusted-third-parties}

Themis~\cite{mengThemisDecentralizedEscrow2019} is a fair exchange protocol that uses blockchain instead of TTP. It provides an escrow service for the secure exchange of cryptocurrencies and digital goods. Also, it provides a decentralized dispute resolution system for resolving conflicts.

The protocol works as follows:
\begin{enumerate}
    \item Alice and Bob generate a 2-of-2 threshold escrow account using Thresh-Key-Gen protocol and send funds to it.
    \item Alice and Bob split their secret keys into \(n=2t+1\) secret shares using Shamir Secret Sharing protocol, where \(n\) is the number of mediators that participate in the decentralized network and \(t+1\) becomes the threshold of a sufficient number of mediators to reconstruct the secret key.
    \item Alice and Bob encrypt each \textit{i}-th key share using public key of \textit{i}-th mediator.
    \item Alice and Bob exchange the sets of encrypted key shares with each other and send funds to the escrow account.
\end{enumerate}

The escrow is secure as long as \(t+1\) of mediators does not collude, which would let them recreate both \(x_A\) and \(x_B\). 

To ensure that parties exchange actual key shares, they send witnesses generated with the Feldman VSS scheme and zero-knowledge proofs to guarantee the consistency between witnesses and key shares.

In case of dispute, the decentralized network of mediators settles the conflict and grand one winning party the other party secret key, allowing it to withdraw the funds.

The monetary incentivization and reputation system guarantee the honesty of mediators.

\subsection{Blockchain, anonymity, and physical delivery}\label{lelantos-a-blockchain-based-anonymous-physical-delivery-system}

Lelantos~\cite{altawyLelantosBlockchainBasedAnonymous2017} is a blockchain-based anonymous
physical delivery system. The protocol achieves anonymity by employing onion routing (similar to the Tor network) to connect physical delivery providers. The whole route from a seller to a buyer is split into multiple steps, and each step is undertaken by the different randomly selected delivery providers. As long as the delivery providers do not collude, the seller neither can learn the identity nor destination address of the buyer.

A smart-contract is used to coordinate the whole process and mediate communication between the buyer and delivery providers.

However, since the system uses Ethereum, it achieves pseudonymity rather than anonymity (see Section~\ref{sec:pseudo-anon}). Also, the protocol does not cover disputes between the parties.

\subsection{Comparision}

We considered only protocols that achieve fair exchange as it is the fundamental feature of such protocols.

Also, we didn't put much attention to the protocols for buying digital products as they are not relevant to our use case. The comprehensive analysis of such protocols is avaiable in~\cite{birjoveanuTwoPartyECommerceProtocols2022}.

Altawy et al., 2017~\cite{altawyLelantosBlockchainBasedAnonymous2017} is a blockchain-based protocol that provides anonymous physical delivery by using onion routing and anonymous blockchain interaction assuming unlinkability between pseudonyms and real identities. However, it does not provide dispute resolution.

Hinarejos et al., 2019~\cite{hinarejosSolutionSecureCertified2019} is the simplest protocol that replaces TTP with blockchain. However, it does not consider anonymity, disputes between parties nor physical material exchange.

Meng et al., 2019~\cite{mengThemisDecentralizedEscrow2019} improves the previous protocol by the crowd-sourced dispute resolution system. However, it does not take anonymity into consideration.

Bîrjoveanu, 2022~\cite{birjoveanuTwoPartyECommerceProtocols2022} is the closest to our protocol, however, it is based on strong assumptions, namely, existence of TTP, banks supporting confidential transactions with commit buffers, and maintaining a global list of coins' serial numbers.

Our protocol achieves anonymity by using either cash or privacy-preserving blockchains. Moreover, our protocol does not require a customer to submit any transaction to the message board, which in other protocols may be the weakest link in achieving the anonymity.

Moreover, none of them directly address our use case. We want the SP to remain public so that the anonymous buyer can easily start a dispute. We assume just one package of physical materials. Moreover, the package is delivered from the buyer to the SP, not the other way around — as it's the case for most e-commerce transactions.

Also, our protocol does not provide its own dispute-resolution mechanism as Themis does. However, we assume the existence of an abstract justice that accepts evidence and punishes the misbehaving party. 
It can be instantiated with either the local court or police, or one of the blockchain
dispute resolution services like Themis~\cite{mengThemisDecentralizedEscrow2019}, Kleros~\cite{bergollaKlerosSociolegalCase2022,gudkovCrowdArbitrationBlockchain2020}, Aragon Court~\cite{aragonDecentralizedDisputeResolution}, LTO Network~\cite{ltonetworkNextGenBlockchainB2B,ltonetworkOnchainIdentitiesCredentials}, and other Online Dispute Resolution platforms~\cite{allenGovernanceBlockchainDispute2019}.


The comparison of the protocols is presented in Table~\ref{tab:comparision}.

The notation \textit{Pseudonymity} means that the anonymity is based on the assumption that the pseudonym is not linked to the real identity; \textit{TTP} means that the protocol uses a trusted third party; \textit{BC} means that the protocol uses a public blockchain; \textit{YES*} means that the protocol provides the feature, but is based on strong or unpractical assumptions.


{\begin{table*}
\centering
\newcommand{\YES}{\cellcolor{green!50}Yes}
\newcommand{\YESBUT}{\cellcolor{green!25}Yes*}
\newcommand{\ID}{\cellcolor{green!25}Identity}
\newcommand{\PSEUDO}{\cellcolor{green!35}Pseudonym}
\newcommand{\ANON}{\cellcolor{green!50}Anonymity}
\newcommand{\NO}{\cellcolor{red!50}No}
\newcommand{\TTP}{\cellcolor{red!50}TTP}
\newcommand{\BC}{\cellcolor{green!50}BC}
\caption{Comparision of related works.}
\label{tab:comparision}
\setlength{\tabcolsep}{3pt}

\begin{tabular}{cccccc}

\noalign{\smallskip}\hline\noalign{\smallskip}
Protocol & Fair exchange & Anonymity & Dispute resolution & Trust & Physical delivery \\
\noalign{\smallskip}\hline\noalign{\smallskip}
\cite{zhangPracticalFairExchangeEPayment2006} (2006) & \YES & \YESBUT & \YES & \TTP & \YES \\
\cite{mohammedalarajFairnessPhysicalProducts2012} (2012) & \YESBUT & \NO & \YES & \TTP & \YES \\
Lelantos~\cite{altawyLelantosBlockchainBasedAnonymous2017} (2017) & \YES & \PSEUDO & \NO & \BC & \YES \\
\cite{hinarejosSolutionSecureCertified2019} (2019) & \YES & \NO & \NO & \BC & \NO \\
Themis~\cite{mengThemisDecentralizedEscrow2019} (2019) & \YES & \NO & \YES & \BC & \NO \\
PPPDCP~\cite{birjoveanuTwoPartyECommerceProtocols2022} (2022) & \YES & \YES & \YES & \TTP & \YES \\
This paper & \YES & \YES & \YES & \BC & \YES \\
\noalign{\smallskip}\hline

\end{tabular}

\end{table*}
 
\section{Building Blocks}\label{sec:building-blocks}

\subsection{Physical products}\label{sec:physical-products}
The use cases we address in this paper use physical materials like blood, urine, hair, and other biological materials.
This requirement, combined with the requirement of anonymity, is a challenging part of the fair-exchange protocols. The problem arises when the seller wants to sent the product to the buyer who wants to remain anonymous. Most of the existing protocols either assume the existence of trusted delivery agent~\cite{mohammedalarajFairnessPhysicalProducts2012,birjoveanuAnonymityFairexchangeEcommerce2015}, or use complex delivery mechanism similar to onion routing involving multiple many delivery services~\cite{altawyLelantosBlockchainBasedAnonymous2017}. 

However, our use case is different in that the physical materials are transferred from the (anonymous) customer to the (public) SP. This, allows for simplification of the delivery process. We assume that there is a way to anonymously deliver a package without revealing the customer personal information either via SP's drop box, Parcel Locker services (e.g. Amazon Locker, InPost~\cite{inpostParcelLockerService}), a customer's trusted party, of even post office.

\subsection{Dispute resolution system}
\label{sec:dispute-resolution}

Disputes are an inevitable part of all human transactions. Whether intentional or accidental, the system should prevent violation of agreed contract rules or local jurisdiction. The rules are specified by law and implemented by the police.

The vision of smart contracts was to replace the legal contract with
programmable and autonomous contracts. The smart contract code encompasses the contract's specifications, hence the slogan \textit{code as a law}. Moreover, smart contracts are automatically executed merging together the gap between law and its enforcement by police~\cite{allenGovernanceBlockchainDispute2019}. However, the blockchain paradigm has its limitations. 
 
Blockchain can assure the correctness of data existing only on the blockchain. The problem arises in the contact between the blockchain and the real world when we want the smart contract to decide based on some input from outside the blockchain. The technique of providing real-world data to the blockchain is called an oracle. Oracle provides data based on a decentralized network of mediators; therefore, the trust is also decentralized~\cite{breidenbachChainlinkNextSteps2021}.

Some oracles provide the data like weather, soccer match result, stock
price, train delay, election results, and others—and they are the ones we
are interested in—provide the settlement of a submitted dispute.

Themis~\cite{mengThemisDecentralizedEscrow2019} besides providing fair exchange protocol,
also offers a semi-autonomous decentralized dispute resolution system that complies with the Web3 postulates of a decentralized web~\cite{ethereumWhatWeb3Why2023}. Themis resolve disputes by a set of voluntary anonymous mediators participating in voting and deciding whether a party misbehaved. The honesty of mediators is achieved by the monetary incentivization and reputation system.

Kleros~\cite{bergollaKlerosSociolegalCase2022} is a smart contract deployed on the Ethereum platform that mimics in a decentralized and autonomous way how the court works in real life. In Kleros, every process of a dispute like gathering evidence, selecting jurors, and rewarding the winning party is automated by a set of smart contracts. Like Themis, the honesty of the agents voting in a case is achieved by game-theoretical economic incentives.

Such a decentralized, voluntary, and anonymous dispute resolution system might work for simple contracts violations like an eBay seller sending broken or wrong products or an Airbnb apartment being inadequate to the photos in the offer. However, it is hard to realize such a decentralized evaluation of health or legal services quality when expertise and privacy concerns are taken into account. Therefore, in our protocol, we take a more conservative approach and resolve disputes by recording evidence on blockchain and then using the local justice system (police or court) to settle the conflict.

Possible directions toward semi-autonomous decentralized resolution system are discussed in Section~\ref{sec:decentralised-justice}.

\subsection{Fairness}\label{fairness}

In case of dispute, the customer can prove to justice (police or court) convincing evidence for the honesty of the customer and the misbehavior of the SP. 
Because the customer is anonymous, the SP can not start a dispute---there is no means to identify the customer.

To mitigate the issue, we designed the protocol so that the SP who follows the protocol is always in an advantaged position; therefore, she has no reason to start a dispute. On the other hand, the customer can start a dispute at any time of the protocol, but only the actual misbehavior of the SP makes him win the case.

Abstracting from the services the SP is providing, each party should be able to prove its honest behavior in case of a dispute. We introduce three proofs that should be disclosed to justice in case of a dispute:

\begin{enumerate}
    \item Proof of delivery ($\mathrm{PoD}$) is a confirmation issued by the SP to the customer that proves that the customer has delivered to the SP a complete (according to the SP requirements) package, and the SP accepted it. It consists of the current time, the deadline to pay for the transaction, the deadline to provide a result of the service, the randomly generated number uniquely identifying the transaction, and the SP's signature guaranteeing non-repudiation. The formal definition of the $\mathrm{PoD}$ is presented in section ~\ref{proof-of-delivery}.
    
    \item Payment $\mathrm{receipt}$ is confirmation that the customer has paid for the transaction at a certain time. The actual implementation depends on the particular cryptocurrency and is discussed further Section~\ref{payment-for-services}.
    
    \item Proof of provision ($\mathrm{PoP}$) is proof that the SP published the result at a certain time. It defends the SP in case the customer unjustly starts a dispute after the result has been published. It consists of a content identifier as specified in section ~\ref{storage-network}, the $\mathrm{nonce}$ uniquely identifying the transaction previously generated in $\mathrm{PoD}$, and the SP's signature guaranteeing authentication. The connection between $\mathrm{PoP}$ and the result is created by the content identifier ($\mathrm{cid}$) that uniquely identifies the result (it is some kind of a hash of the result) such that the result can not be forged after the $\mathrm{PoP}$ has been published.
\end{enumerate}

\subsection{Message Board}\label{sec:message-board}
The proof of provision that we coined for the purpose of this protocol is generally called the proof of existence~\cite{crespoStamperyBlockchainTimestamping2017}.

The idea behind proof of existence is to certify that some information
has existed at a certain moment, in such a way that nobody can
undermine its existence, integrity, and ownership (also called the
non-repudiation requirement).


We need this functionality for two reasons: (1) communicate the existence of the result to the customer to whom the SP has no other communication medium as the customer stays anonymous; (2) let the SP proof the publication of the result within the agreed with customer deadline. 

By publishing the $\mathrm{PoP}$ on the blockchain, the SP can not falsify the time at which the result has been provided because the block creation time proves that. Blockchain acts as a global clock that securely timestamps everything that gets into the block, so the $\mathrm{PoP}$ that was included in a block will be associated with the time when the block has been created. Moreover, since the blockchain is public, everyone (including justice) can be convinced that the SP indeed published the result at that time.

Without such proof, there would be no other means to settle the conflict between the customer claiming the result has not been published and the SP claiming the result has been published within the deadline.

Depending on the context the platform for achieving it is called bulletin board~\cite{achenbachImprovedCoercionresistantElectronic2015}, trusted timestamping~\cite{gippDecentralizedTrustedTimestamping2015}, or message board~\cite{hinarejosSolutionSecureCertified2019}. In this work, we call it a message board.

We keep the protocol general enough to be implemented using any existing technology to provide message board service, assuming it is decentralized and supports subscribing for the upcoming proofs from a particular address.

\subsection{Anonymity, pseudonomity, and confidentality}\label{sec:pseudo-anon}

Privacy is a concept used in almost all social sciences like philosophy, psychology, sociology, and legal. The multidisciplinary nature leads to ambiguous definitions~\cite{smithInformationPrivacyResearch2011}. For our work, we rely on more concrete definitions, i.e.~confidentality, and anonymity.

Confidentiality is the ability to hide the details of action from others. Alternatively, we can say that the system guarantees confidentiality if, for all observers, everything they can say about the action is the fact that it happened and nothing more.

Anonymity is the ability to hide one identity from others. More precisely, the inability to correlate actions performed within the system with the user's identity. Alternatively, we can say that the system guarantees anonymity if, for all observers, the actions are equally likely to be associated with any user of the system. However, anonymity is a spectrum rather than a dichotomous classification. One method of quantifying the level of anonymity is \textit{k}-anonymity proposed in~\cite{sweeneyKanonymityModelProtecting2002}. It measures the user's anonymity by the number of other users from whom the user can not be distinguished. Concretely, the user is \textit{k}-anonymous if his actions are equally likely associated with \textit{k}-1 other users; the larger \textit{k}, the higher anonymity.

Some anonymity techniques can be deployed on non-anonymous blockchains. The so-called mixers gather users into an \textit{anonymity set}, which then collude together to launder transactions in such a way that to an observer, the likelihood of the sender of each transaction is equiprobable for any user from the anonymity set.

Some systems guarantee pseudonymity rather than anonymity. Pseudonymity allows users to hide their real identity behind pseudonyms. Despite the whole system being transparent and allowing linking actions to pseudonyms, the system is considered anonymous as long as the link between pseudonyms and real identities is secret. This assumption is hard to satisfy in practice, as the KYC (Know Your Customer) and AML (Anti Money Laundering) regulations require users to reveal their real identity to cryptocurrency exchanges, making the user's anonymity dependent on the security of the IT system. Moreover, some correlations can be inferred merely by the analysis of transactions \cite{androulakiEvaluatingUserPrivacy2013, oberStructureAnonymityBitcoin2013}.

Figure ~\ref{fig:anonymity-diagram} illustrates relations between these terms.

\begin{figure}[h!]
\includegraphics[width=\linewidth]{anonymity-diagram.png}
\centering
\caption{Let us assume Alice to be the customer willing to keep her identity anonymous and Bob to be the public SP. Alice controls two addresses, 1 and 2; the connection between her real identity and the first address has been compromised, and therefore the identification is possible; the connection between the second pseudonym is still unknown and therefore anonymous. Alice takes two actions; the first one from the compromised address and the second one from the anonymous one. The first action is confidential; therefore, even though the pseudonym has been compromised, the action can not be associated with Alice. The second action is transparent; therefore, Alice maintains her anonymity as long as the connection to the second pseudonym is concealed.}

\label{fig:anonymity-diagram}
\end{figure}

The privacy-preserving blockchains are the ones that maintain anonymity via untraceability and (ideally) unlikability, not via the assumption that the connection between an address (pseudonym) and the real identity is concealed.   

Examples of blockchains that natively support confidential transactions are: Monero~\cite{vansaberhagenCryptoNote2013} (using Ring Signatures~\cite{noetherRingSignatureConfidential2015} or Bulletproofs~\cite{bunzBulletproofsShortProofs2018}), ZCash~\cite{ben-sassonZerocashDecentralizedAnonymous2014} (using zkSNARK~\cite{ben-sassonSNARKsVerifyingProgram2013}), Grin~\cite{fuchsbauerAggregateCashSystems2019} (using Mimblewimble~\cite{jedusorMIMBLEWIMBLE2016}), and IronFish~\cite{ironfishPrivateAnonymousEasy} (using Sapling protocol~\cite{hopwoodZcashSaplingProtocol2022}).

Overly techniques that achieve anonymity on top of non-privacy-preserving blockchains are: Ethereum's Tornado Cash~\cite{pertsevTornadoCashPrivacy2019} (via zkSNARK~\cite{grothSizePairingbasedNoninteractive2016} and MiMC~\cite{albrechtMiMCEfficientEncryption2016}), Bitcoin's Wasabi~\cite{wasabiwalletBitcoinPrivacyWallet} (via CoinJoin~\cite{maxwellCoinJoinBitcoinPrivacy2013}).

\subsection{Payment for services}\label{payment-for-services}
Transactions between customers and SPs must be pegged to prevent reusing one payment for multiple transactions. 
In other words, we need a mechanism that uniquely associates a payment with the corresponding transaction.

Depending on the cryptocurrency, the link can be created in different
ways:

\begin{itemize}

\item separate address: each transaction uses an unique address associated with the transaction. Such addresses can be derived using Hierarchical Deterministic Wallets~\cite{wuilleBIP32HierarchicalDeterministic2012} and published on message board, to achieve non-repudiation.
\item memo: the payments are sent to sigle SP account, but contain an extra field named ``memo'' filled with the unique indentifier $\textrm{provisionID}$. 
\end{itemize}

Each payment containing the $\textrm{provisionID}$ in the memo, or payment to the dedicated address is considered to pay for the transaction. 

In case of a dispute, there must be a way to prove to justice that the customer has paid for the transaction. As the proof of payment is trivial in transparent and trackable blockchains, it gets more complicated when it comes to anonymous blockchains. Monero allows proving and checking payments via dedicated API~\cite{moneroHowProvePayment}. ZCash provides a mechanism called Payment Disclosure~\cite{daviesIntroductionPaymentDisclosure2017}. We call the proof of payment a \textit{payment recepit}.

\subsection{Storage network}\label{storage-network}
Once the SP finishes its service, she has to provide the result to the customer. The most natural approach would be to send the result via e-mail or some dedicated platform. However, the customer wants to stay anonymous and so does not want to expose his e-mail address nor IP address. Moreover, the SP should prove that the result has been provided before the deadline, which brings us to the issue of proof of existence discussed in Section ~\ref{sec:message-board}.

One approach would be to post the result into a blockchain. However, storing data on a blockchain is very expensive. The most common workaround (\cite{shahidBlockchainBasedAgriFoodSupply2020, wangAuditableProtocolsFair2019, chenImprovedP2PFile2017}) is to publish the data on a content addressable peer-to-peer storage network like IPFS~\cite{benetIPFSContentAddressed2014}. Then, publish on the blockchain just the content identifier ($\mathrm{cid}$) that uniquely points to the content stored on IPFS.

We take the same approach. Once the SP creates the result, she encrypts it using the previously provided encryption key and uploads it to the IPFS network.
%TODO: write about DHKE encyption

To increase anonymity, the customer should use standard techniques to hide its IP address, such as VPN, or proxy.

\subsection{Separation of concerns}
We could use one blockchain to achieve all of these three roles: (i) anonymous payments, (ii) message board, (iii) storage network.

While most blockchains could provide message board functionality, anonymous payments are not as common. Especially a provable storage network is a functionality of a few specialized blockchains.

Instead of searching for one blockchain that provides all the functionalities, we allow the protocol to use separate blockchains for each role. If a suitable blockchain arises, it can play more than one role.

At the time of writing, we see the following technologies that fulfill the requirements of each role:

\begin{enumerate}
\def\labelenumi{\arabic{enumi}.}

\item Anonymous payments: Monero \cite{vansaberhagenCryptoNote2013}, ZCash
  \cite{ben-sassonZerocashDecentralizedAnonymous2014}, Grin \cite{fuchsbauerAggregateCashSystems2019},
  Tornado Cash \cite{pertsevTornadoCashPrivacy2019}.
\item Message board: Open timestamps~\cite{opentimestampsTimestampingProofStandard}, Stampery \cite{crespoStamperyBlockchainTimestamping2017}, Bitcoin blockchain (Proof of Existence~\cite{proofofexistenceWebApplicationProve}, Chainpoint~\cite{chainpointBlockchainProofAnchoring}), Ethereum blockchain, or any other public blockchain that supports attaching extra data along the transaction.
\item Storage network: IPFS~\cite{benetIPFSContentAddressed2014}, Filecoin~\cite{protocollabsFilecoinDecentralizedStorage2017}, or Ethereum's
  Swarm~\cite{teamSWARMStorageCommunication2021}.
\end{enumerate}

\section{The Protocol}\label{sec:protocol}
In this section, we propose an abstract protocol for anonymous service provisioning without any assumptions about the underlying technologies. We define the requirements for each role and leave the choice of the technology to the developer. Later in the paper (Section~\ref{sec:experiments}) we describe our implementation used to conduct an experiment.

\subsection{Assumptions}

\begin{itemize}
\item There exists PKI infrastructure:
    \begin{itemize}
        \item The customer and the SP have their key pairs consisting of secret key $\mathrm{sk}(\mathrm{party})$ and public key $\mathrm{pk}(\mathrm{party})$, where $\mathrm{party} \in \{\mathrm{C}, \mathrm{SP}\}$ for the customer and the SP accordingly.
        \item Both the customer and the SP can create and verify digital signatures created by the customer $\mathrm{sig}_{\mathrm{sk}(\mathrm{C})}$ and the SP $\mathrm{sig}_{\mathrm{sk}(\mathrm{SP})}$.
        \item The SP's public key $\mathrm{pk}(\mathrm{SP})$ is publicly known.
    \end{itemize}
    
\item Both the customer and the SP:
    \begin{itemize}
        \item use common symmetric encryption $\mathrm{E}_\mathrm{key}(\cdot)$ and decryption $\mathrm{D}_\mathrm{key}(\cdot)$ operations.
        \item have access to anonymous payments blockchain, message board, and storage network.
    \end{itemize}

\item The SP:
    \begin{itemize}
        \item accepts packages from unknown customers.
        \item accepts payments with cash and anonymous cryptocurrencies as described in Section~\ref{payment-for-services}.
    \end{itemize}
    
\item Justice:
    \begin{itemize}
        \item accepts as evidence in a dispute the $\mathrm{PoD}$, $\mathrm{PoP}$, and payment $\mathrm{receipt}$ as described in Section~\ref{fairness}.
    \end{itemize}

\item Anonymous payments blockchain:
    \begin{itemize}
        \item supports anonymous, i.e., untraceable and (ideally) unlinkable transactions as specified in Section~\ref{sec:pseudo-anon}.
        \item supports uniquely identifying transactions via a dedicated address, memo field, or other similar mechanisms as described in Section ~\ref{payment-for-services}. 
    \end{itemize}

\item Message Board:
    \begin{itemize}
        \item supports transactions of sizes up to $\mathrm{PoD}$ and $\mathrm{PoP}$.
    \end{itemize}

\item Storage network:
    \begin{itemize}
        \item allows for content retrieval via content identifier $\mathrm{cid}$ (usually a hash of the content).
        \item allows for anonymous content retrieval.
        \item guarantee that the content will be available for the duration of the agreement.
    \end{itemize}
\end{itemize}

\subsection{Messages}\label{messages}
In this section, we describe the messages exchanged between the parties of the protocol.

\vspace{5mm}

\noindent \textbf
{Package}\label{package} is a physical container prepared by the customer encompassing all $\mathrm{materials}$ required by the SP to provide the service.

$$\mathrm{pkg} \equiv (\mathrm{materials}, \mathrm{provisionID}, \mathrm{pk(C)})$$

where:

\begin{itemize}

\item $\mathrm{materials}$ - are the materials required to provide the service, for example, samples of urine, blood, stool, saliva; legal documents, CDs, emails, photos, bank statements; or any other kind of materials depending on the service.
\item $\mathrm{provisionID}$ - a randomly generated provision identifier, used to anonymously track the provision throughout the protocol's steps.
\item $\mathrm{pk(C)}$ - customer's public key used to encrypt the result published on the public storage network.
\end{itemize}

\noindent \textbf
{Proof of delivery ($\mathrm{PoD}$)}\label{proof-of-delivery} is an attestation to the fact that the customer has delivered a correct (according to the SP requirements) package to the SP, and the SP has accepted it.

It is also an agreement between the customer and the SP, as it includes agreed upfront deadlines of actions and a payment method.

$\mathrm{PoD}$ is published on the message board by the SP.

\begin{eqnarray}
\mathrm{PoD} & \equiv & (\begin{array}[t]{l}\mathrm{T}_\mathrm{issue}, \mathrm{T}_\mathrm{pay}, \mathrm{T}_\mathrm{provide}, \\\\ \mathrm{address}, \mathrm{provisionID}, \mathrm{pk(C)}, \mathrm{sig}_\mathrm{SP} \; )\end{array}
\end{eqnarray}

where:

\begin{itemize}

\item $\mathrm{T}_\mathrm{issue}$ - time at which the $\mathrm{PoD}$ is issued by the SP.
\item
  $\mathrm{T}_\mathrm{pay}$ - deadline to pay for the transaction.
\item
  $\mathrm{T}_\mathrm{provide}$ - deadline to provide the result of the service, i.e. publish $\mathrm{PoP}$.
\item $\mathrm{address}$ - payment address of the SP's anonymous blockchain account.
\item $\mathrm{provisionID}$ - the number uniquely identifying the transaction previously generated by the customer.
\item $\mathrm{pk(C)}$ - customer's public key used to encrypt the result published on the public storage network.
\item $\mathrm{sig}_\mathrm{SP}$ - the SP's signature guaranteeing non-repudiation.
\end{itemize}

also:
\(\mathrm{T}_\mathrm{issue} \leq \mathrm{T}_\mathrm{pay} \leq \mathrm{T}_\mathrm{provide}\)

\noindent \textbf
{Proof of provision ($\mathrm{PoP}$)}\label{proof-of-provision} is proof that the SP published the result at a certain time. It protects the SP in case the customer unjustly starts a dispute after the result has been published. The connection between $\mathrm{PoP}$ and the result is made by the content identifier ($\mathrm{cid}$) uniquely identifying the result such that the result can not be forged after the $\mathrm{PoP}$ has been published.

$\mathrm{PoP}$ is published on the message board by the SP.


\begin{eqnarray}
\mathrm{PoP} & \equiv & (\mathrm{cid}, \mathrm{provisionID}, \mathrm{sig}_\mathrm{SP})
\end{eqnarray}

where:

\begin{itemize}

\item $\mathrm{cid}$ - content identifier as specified in Section~\ref{storage-network}.
\item $\mathrm{provisionID}$ - the number uniquely identifying the transaction previously generated by the customer.
\item $\mathrm{sig}_\mathrm{SP}$ - the SP's signature guaranteeing non-repudiation.
\end{itemize}

\noindent \textbf
{Payment receipt}\label{payment-receipt} proves that the customer did the payment. Since the evidence depends on a particular blockchain (see Section~\ref{payment-for-services}), we refer to it symbolically as
$\mathrm{receipt}$.

\noindent \textbf
{Result}\label{results} is assumed to be a document in PDF format, but any other format is allowed as long as it can be binary encoded and uploaded to the storage network. We refer to it symbolically as $\mathrm{result}$.

\noindent \textbf
{Content Identifier (cid)}\label{content-identifier-cid} is a term coined by IPFS~\cite{ipfsContentIdentifiersCIDs}. However, since our protocol does not depend on this particular implementation of the storage network, we let the $\mathrm{cid}$ be any other identifier that securely and uniquely points to the content.

\subsection{Protocol description}\label{protocol-description}

In this section, we describe each step of the protocol, also shown in the Figure~\ref{fig:protocol-diagram}.

\begin{figure}[ht!]
\includegraphics[width=\linewidth]{anonser-protocol.png}
\centering
\caption{Messages exchanged in the protocol. Solid arrows indicate requests and dashed arrows indicate responses.}
\label{fig:protocol-diagram}
\end{figure}

\noindent \textbf
{Step 0.  Preparation}\label{step-0-preparation}

The customer collects all the $\mathrm{materials}$ required by the SP, generates a random $\mathrm{provisionID}$ and a random keypair $(\mathrm{sk(C)},\mathrm{pk(C)})$. The $\mathrm{provisionID}$ will be used to associate all actions related to the transaction throughout the protocol. The $\mathrm{provisionID}$ and $\mathrm{pk(C)}$ are encoded as a QR code, printed, and stuck to the package $\mathrm{pkg}$. The $\mathrm{sk(C)}$ is kept secret and used to decrypt the $\mathrm{result}$ at the end of the protocol.

\noindent \textbf
{Step 1. Package delivery}\label{step-1-package-delivery}

The protocol starts once the customer delivers package $\mathrm{pkg}$ to the SP and its content gets accepted. As a result, $\mathrm{PoD}$ is created with predefined payment deadline $\mathrm{T}_\mathrm{pay}$, service provision deadline $\mathrm{T}_\mathrm{provide}$, and current time $\mathrm{T}_\mathrm{issue}$. Also, $\mathrm{PoD}$ embodies the information whether the service has been paid in cash or should be paid by the customer using the anonymous blockchain account. In the latter case, the SP's payment $\mathrm{address}$ is included in the $\mathrm{PoD}$.

The digital signature $\mathrm{sig}_{\mathrm{sk}(\mathrm{SP})}$ on the $\mathrm{PoD}$ created with the SP's secret key $\mathrm{sk}(\mathrm{SP})$ guarantee non-repudiation.

Symbolically: 
\[
\mathrm{PoD \gets delivery(pkg)}
\]

\noindent \textbf
{Step 2. Proof of Delivery}\label{step-2-pod}

Then, the $\mathrm{PoD}$ is published on the message board by the SP, committing to the fact that the package $\mathrm{pkg}$ has been delivered, and the SP can not reject receiving it. If the provision has not been paid in cash, the SP waits for the customer to pay for the service on the specified in the $\mathrm{PoD}$ payment $\mathrm{address}$.

Symbolically: 
\[
\mathrm{publish(PoD)}
\]

\noindent \textbf
{Step 3. Get Proof of Delivery}\label{step-3-get-pod}

Once the package is delivered and the $\mathrm{PoD}$ is published, the customer can get the $\mathrm{PoD}$ from the message board and (if everything is correct) proceed with the protocol.

Symbolically: 
\[
\mathrm{PoD \gets get(provisionID)}
\]

\noindent \textbf
{Step 4. Payment}\label{step-4-payment}

If the provision has not been paid in cash, the customer should pay for the transaction with the predefined anonymous payment blockchain (see Section~\ref{payment-for-services}).
In return, the customer receives the $\mathrm{receipt}$ that should be disclosed in case of a dispute.

Symbolically: 
\[
\mathrm{receipt \gets payment(address)}
\]

\noindent \textbf
{Step 5. Provision of service}\label{step-5-provision-of-service} 

Once the customer has paid for the transaction either in cash or using the anonymous blockchain, the SP can start providing the service.

Symbolically: 
\[
\mathrm{result \gets provision(materials)}
\]

\noindent \textbf
{Step 6. Upload result}\label{step-6-upload-result}

After the service is finished, a result should be created. 
Next, the result is encrypted using a shared key derived from the customer's public key $\mathrm{pk(C)}$ and the SP's secret key $\mathrm{sk(S)}$ using Diffie-Hellman key exchange (DHKE) method~\cite{diffieNewDirectionsCryptography1976}.
The encrypted result is then uploaded on the content addressable network (such as IPFS). In return, the content identifier ($\mathrm{cid}$) is created.

Symbolically: 
\[
\mathrm{cid \gets upload(E_{DHKE(sk(SP), pk(C))}(result))}
\]

\noindent \textbf
{Step 7. Proof of provision}\label{step-7-proof-of-provision}

When the $\mathrm{result}$ is uploaded, the SP creates and publishes on a message board a proof of provision consisting of $\mathrm{cid}$ along with $\mathrm{provisionID}$ and signature $\mathrm{sig}_\mathrm{SP}$.

Symbolically: 
\[
\mathrm{publish(PoP)}
\]

\noindent \textbf
{Step 8. Get Proof of Provision}\label{step-8-get-proof-of-provision}

After delivering the package and paying for the transaction, the customer starts listening to the message board and waits until the SP publishes the $\mathrm{PoP}$ for the $\mathrm{provisionID}$.

Symbolically: 
\[
\mathrm{cid \gets get(provisionID)}
\]

\noindent \textbf
{Step 9. Download result}\label{step-9-download-result}

Having the $\mathrm{cid}$, the customer downloads and decrypts the $\mathrm{result}$ using a shared key derived from the customer's secret key $\mathrm{sk(C)}$ and the SP's public key $\mathrm{pk(S)}$ using DHKE method. The protocol ends.

Symbolically: 
\[
\mathrm{result \gets D_{DHKE(sk(C), pk(SP))}(download(cid))}
\]

\section{Fairness analysis}\label{sec:fairness-analysis}
We analyze the fairness of the protocol by representing it as an interactive non-cooperative game.

\subsection{Model}\label{sec:fairness-model}
We consider three positions:

\begin{itemize}
\item Neutral position (•): when a party has not spent nor gained anything of significant value (money, time, effort). For example, at the beginning of the protocol.
\item Disadvantaged position (-): when a party has put a significant value without receiving an equivalent. For example, the customer has paid for a service in advance.
\item Advantaged position (+): when a party would benefit if the transaction would halt at that step. For example, the SP has received payment before service provision.
\end{itemize}

There are many actions that each party can take, but we group them into two categories:

\begin{enumerate}
\def\labelenumi{\arabic{enumi}.}

\item Normal: taking actions prescribed by the protocol.
\item Abnormal: everything that deviates from the designed steps of the protocol. For example, sending an arbitrary message, skipping or repeating steps, and timing out.
\end{enumerate}

Moreover, at any step of the protocol, the customer can start a dispute; therefore, another dimension with two positions has to be considered:

\begin{enumerate}
\def\labelenumi{\arabic{enumi}.}

\item Agree: the customer agrees with the action and therefore does not start a dispute.
\item Start a dispute: the customer disagrees with the action and therefore starts a dispute.
\end{enumerate}

As a result, in our analysis, we have to consider four different outcomes for each party of the protocol ($\mathrm{party \in \{c, s}\}$), for each step of the protocol ($\mathrm{step \in 1..9}$):

\begin{itemize}

\item
  $\mathrm{\sigma_{step,party,n}}$: after following the protocol when the other party acted normally.
\item
  $\mathrm{\sigma_{step,party,d}}$: after a settled dispute when the other party acted normally.
\item
  $\mathrm{\sigma_{step,party,\overline{n}}}$: after not starting a dispute despite the other party has acted abnormally.
\item
  $\mathrm{\sigma_{step,party,\overline{d}}}$: after a settled dispute when the other party has acted abnormally.
\end{itemize}



The protocol terminates after the last step, after starting a dispute, or after a party has not completed its designated action in time. Therefore, all positions but $\mathrm{\sigma_n}$ are termination positions.

Because the customer is anonymous, the SP can not start a dispute—there is no means to identify the customer. To mitigate the issue, we designed the protocol so that the SP who follows the protocol is always in an advantaged position, and therefore has no reason to start a dispute. On the other hand, the customer can start a dispute at any time of the protocol, but only the actual misbehavior of the SP makes him win the conflict.

\begin{definition}[Fairness] \label{def:fairness}
A protocol achieve fairness iff 
\begin{equation*}
\begin{split}
\forall_{party \in parties}\forall_{\mathrm{step} \in \mathrm{steps}} &\operatorname{can\ move}\\
&\operatorname{to\ the\ non-disadvantaged\ position} 
\end{split}
\end{equation*}

\end{definition}


\subsection{Assumptions}\label{sec:assumptions}

Below we list the assumptions we take for the analysis purpose.  

\begin{enumerate}
\item Both parties start from a neutral position $\neutral$.
\item After completing a transaction, both parties end up in advantaged positions (\plus{}). In other words, they have an intrinsic motivation to initialize and complete the transaction.
\item The protocol steps are atomic—there are no intermediate steps.
\item The protocol can go only forward—there is no way of reverting any action.
\item Repeating the first step starts a new transaction. Repeating any other step is considered abnormal and gets ignored. For example, paying for the invoice twice does not cause any effect on the curse of the protocol.
\item Once published, the result is available to the customer on the storage network. The idea of guaranteeing it cryptographically is discussed in Section~\ref{sec:discussion}.
\item Winning the dispute leads to a neutral position ($\neutral$).
\item Losing the dispute leads to a punishment greater than any reward, leading to a disadvantaged position (\minus{}). Hence, the rational customer will not start a dispute that he is unsure of winning.
\item Both the customer and the SP are rational (selfish). They always prefer to go from the worst position (\minus{} or $\neutral${}) to a better position ($\neutral$ or \plus), but also risk a temporary worst position in favor of a later better position if and only if at any time they can get out of the worst position; in particular, at any time the customer can start a dispute and, if the SP has misbehaved, get out of disadvantaged position (\minus) to the neutral position ($\neutral$).
\item We assume that the Customer's materials without personal information are useless, and the effort to deliver the package is negligible, so that the Customer's first step does not lead to disadvantaged position.
\item We assume the processing cost of publishing $\mathrm{PoD}$ is negligible and balance out the effort of delivering the package by the Customer.
\end{enumerate}

\subsection{Steps}\label{sec:steps}

Figure~\ref{fig:positions} shows the positions of each party after each possible action taken.

\begin{figure}[h!]
\includegraphics[width=\linewidth]{model.png}
\centering
\caption{Positions after each step of protocol}
\label{fig:positions}
\end{figure}

The detailed description of each step and the justification
of the outcome positions is given in Appendix~\ref{app:proof-of-fairness}.

\subsection{Example scenarios}\label{example-scenarios}

The Figure~\ref{fig:misbehaviour} shows the transitions of positions when the SP tries to misbehave and does not execute service, and so does not publish $\mathrm{PoP}$ after receiving the payment. After the deadline $\mathrm{T}_\mathrm{provide}$ the SP can not publish valid $\mathrm{PoP}$ and so the Customer starts a dispute. The Customer wins the dispute as the SP is unable to proof the $\mathrm{PoP}$ was published before $\mathrm{T}_\mathrm{provide}$. The protocol ends up neutral to the customer and disadvantaged to the SP positions.

\begin{figure}[h!]
\includegraphics[width=\linewidth]{misbehaviour.png}
\centering
\caption{Transitions of positions where the SP is misbehaving and the customer starts a dispute}
\label{fig:misbehaviour}
\end{figure}
the rational path for both parties is to follow the protocol as shown in the Figure~\ref{fig:rational}.

\begin{figure}[h!]
\includegraphics[width=\linewidth]{rational.png}
\centering
\caption{Transitions of positions where both the customer and the SP are following the protocol}
\label{fig:rational}
\end{figure}

Using the fairness definition~\ref{def:fairness} and the assumptions stated in Section~\ref{sec:assumptions}, our analysis indicates that the protocol achieves fairness.

\section{Experiments}\label{sec:experiments}

\subsection*{Setup}

We have developed a prototype of our protocol using the following technologies:

\begin{itemize}
  \item{Anonymous payments} — we use the Monero blockchain \cite{noetherRingSignatureConfidential2015}.
  \item{Storage netowork} — we use the Powergate~\cite{textilePowergate2023}, which is a wrapper around Filecoin and IPFS.
  \item{Message board} — we use the Ethereum blockchain \cite{woodEthereumSecureDecentralised2014}; concretely, its local development version Truffle Ganache and Solidity language.
  \item{Customer and SP} — we create client side web application (webapp) using \texttt{React.js} for creating UI, and \texttt{web3.js} library for interaction with Ethereum. We use \texttt{MetaMask} browser extension for signing and submitting transactions to Ethereum's node. We use \texttt{monerod} network client and \texttt{monero-wallet-cli} command-line interface for interacting with Monero blockchain.
\end{itemize}

The prototype is available at [blinded for review process]. The source code is available at [blinded for review process].

% The prototype is available at \url{https://anonser.stan.bar}. The source code is available at \url{https://github.com/stanbar/anonymous-provision-of-services-via-blockchain}.

We developed and tested our prototype in the following environment: OS — Arch Linux 5.11.8; Docker — 20.10.5; CPU — Intel(R) Core(TM) i7-4790 (8) 4.00~GHz; RAM — 16 GB DDR3; Storage — Samsung PM85 256~GB SSD; monerod and monero-wallet-cli — v0.18.1.2; Powergate — v2.6.2; Ganache — v7.5.0; Solidity — v0.8.17; ReactJS — v18.0.25; web3.js — v1.8.1; crypto-js — v4.1.1.

For simplicity, all components run on one, mentioned above, physical machine; and all processes are managed by Docker. 

Moreover, Powergate is configured to use local Filecoin and IPFS networks.
For Ethereum blockchain, we use Truffle Ganache, which is a local Ethereum blockchain for development and testing purposes. 
Monero is configured to use the public stage network.
We assume the service provider offers only one type of service, which is offered for a fixed public price, hence we omit the type of service and price from the protocol.

\subsection*{Preparation}

Both the customer and the SP create Monero wallets using \texttt{monero-wallet-cli} command-line tool.

The service provider deploys the smart contract using the \texttt{truffle migrate --network development} command, which deploys the smart contract to the Ethereum blockchain. The webapp is configured to use the latest deployed smart contract address.

The customer gets some testing Monero funds using faucet service available at \url{https://community.rino.io/faucet/stagenet/}.

Next, the customer enables per-transaction proof generation (payment \texttt{receipt}) by setting \texttt{set store-tx-info 1} for his wallet. Proof generation is required to verify the payment in case of a dispute.

At this point, both the customer and the SP are ready to start the protocol.

\subsection*{Experiment}

The customer and the SP are two different users of our prototype, but for simplicity, they use the same machine and the same web application.

Figures \ref{fig:anonser-experiment1} and \ref{fig:anonser-experiment2} show the steps of the experiment. Their description is as follows:

\begin{enumerate}
  \setcounter{enumi}{0}
  \item[0.] The protocol starts with the customer opening the webapp and creating a new provision. 
The app generates a random ECDSA (secp256k1) customer's keypair and random 32 bytes provisionID, then display QR code that encodes both the provisionID and the customer's public key. 
The customer's private key must be downloaded and provided later for the decryption of the results.

  \item[1.] The customer prints the QR code, sticks it on a package and delivers the package to the SP either: in person, via a trusted party, or a delivery agency.

  \item[2.1.] The SP opens the app, and scans the QR code decoding the provisionID and the client's public key.

  \item[2.2.] Since (in this experiment) the provision has not been paid in cash, the SP generates a unique Monero payment address using \texttt{monero-wallet-cli integrated\_address} 
  \item[2.3.] The SP submits the proof of delivery to the Ethereum blockchain using MetaMask interface. 

  % monero-wallet-cli --stagenet --wallet-file sp --password "" integrated_address
  % Random payment ID: <2b6d65d7d48b7896>
  % Matching integrated address: 5LHmrsVsQM2Q2TgqufX4A5gKPpxy1czHULDyVc84omvnh1nQLVmqXk4VBuy8WnX1AXfKxDx7xuASAc6svkZqGVkL1XkUBWT7QejHymEaYM
  \item[3.] The customer (using the webapp) checks the transaction status on the Ethereum blockchain invoking \texttt{getProvision} with arguments \texttt{customerPubKey} and \texttt{provisionID}.
  % monero-wallet-cli --stagenet --wallet-file customer --password "" transfer 5LHmrsVsQM2Q2TgqufX4A5gKPpxy1czHULDyVc84omvnh1nQLVmqXk4VBuy8WnX1AXfKxDx7xuASAc6svkZqGVkL1XkUBWT7QejHymEaYM 1
  % Transaction successfully submitted, transaction <7d89c04de458cfb76a811d5eb325075dec59f7a993c3bf7ce37f9e3a1630af65>
  % monero-wallet-cli --stagenet --wallet-file customer --password "" get_tx_key 7d89c04de458cfb76a811d5eb325075dec59f7a993c3bf7ce37f9e3a1630af65
  % Tx key: 1364dc848a752bf52011f4a63d98bcb16091cacd985713b6b5264b3ede6de40f

  \item[4.] The customer (using the \texttt{monero-wallet-cli transfer}) sends the payment to the designated in the smart contract \texttt{paymentAddress} and stores the payment receipt (using \texttt{monero-wallet-cli get\_tex\_key <tx-id>}) in case of a dispute.

  \item[5.] Once SP notices the payment on Monero blockchain it starts providing the service and outputs the file \texttt{result.pdf}.

  \item[6.] The SP uploads the file \texttt{result.pdf} on the Filecoin and IPFS networks using Powergate. As a result, the SP gets the content identifier \texttt{cid}, and \texttt{dealID} and \texttt{minerID}.

  \item[7.] The SP submits a Proof of Provision transaction to the Ethereum blockchain by invoking \texttt{proofOfProvision} with arguments \texttt{customer\-PubKey}, \texttt{provisionID}, \texttt{cid}, and \texttt{dealID}.

  \item[8.] Meantime, the Customer subscribes to Ethereum and waits until the SP publishes the Proof Of Provision.
 Once the Proof Of Provision is published the Customer downloads the result using either: IPFS network via \url{https://dweb.link/<cid>} or Lotus network using \texttt{lotus retrieve <cid> <minerID>}. 
 
  \item[9.] Results are then decrypted using the previously stored customer's private key. If everything is correct, the Customer is satisfied with the service and the protocol ends, otherwise, the Customer can initiate a dispute.

\end{enumerate}



\ifx\FORMAT\SINGLECOLUMN
\noindent%
\begin{minipage}{\linewidth}
\makebox[\linewidth]{
\includegraphics[height=0.95\textheight,keepaspectratio]{anonser-experiment1.pdf}}
\captionof{figure}{Steps of the experiment, first part.}\label{fig:anonser-experiment1}
\end{minipage}

\noindent%
\begin{minipage}{\linewidth}
\makebox[\linewidth]{
\includegraphics[height=0.95\textheight,keepaspectratio]{anonser-experiment2.pdf}}
\captionof{figure}{Steps of the experiment, second part.}\label{fig:anonser-experiment2}
\end{minipage}
\else
\begin{figure*}
  \centering
  \includegraphics[height=0.95\textheight,keepaspectratio]{anonser-experiment1.pdf}
  \caption{Steps of the experiment, first part.}\label{fig:anonser-experiment1}
\end{figure*}

\begin{figure*}
  \includegraphics[height=0.95\textheight,keepaspectratio]{anonser-experiment2.pdf}
  \caption{Steps of the experiment, second part.}\label{fig:anonser-experiment2}
\end{figure*}

\fi
\subsection{Results}

\paragraph{Fariness}
As shown in Section~\ref{sec:steps} and~\ref{app:proof-of-fairness}, 
the protocol is fair. It is achieved by an undeniable hand-shake mechanism in which the SP first publishes $\mathrm{PoD}$ (step 2.) committing to package delivery and deadlines of the service, and then the customer accepts it by paying for the service or not (step 3.).

Once the payment is made, the SP stays in the advantaged position. Since he has agreed on the deadlines of the provision of the service, he is incentivized
to provide the service and publish the results and PoP before the deadline (step 7.); otherwise the customer—having all the evidence—can initiate a dispute and punish the SP, therefore rational parties will follow the protocol.

Non-repudiation without TTP is achieved by using the blockchain and digital signatures. Blockchain guarantees that each change to the state of the smart contract is visible to everyone and only the SP is allowed to modify its state. 

\paragraph{Anonymity}
We achieved anonymity by using anonymous payment methods such as cash or privacy-preserving payment blockchains (e.g., Monero), and decentralized storage networks like IPFS and Filecoin, allowing the customer to interact with the protocol without revealing his identity at any step of the protocol.

\paragraph{Provable Results Availability}
Content addressable networks such as IPFS can not guarantee the availability of the content\footnote{Nothing prevents the SP from publishing the result, receiving the $\mathrm{cid}$, publishing $\mathrm{PoP}$, and immediately after removing the result from the local storage. In case of dispute, the SP can upload the content again, proving its availability. The SP has no motivation to proceed with this kind of misbehavior other than putting the customer in a disadvantaged position caused by the lost dispute.}. However, the Filecoin~\cite{protocollabsFilecoinDecentralizedStorage2017}, which works as an incentivization layer on top of IPFS, increases the content availability via economic incentivisation (i.e., punishment in case of lack of proof of storage of the content~\cite{filecoinSlashing}). Compared to typical blockchain networks, Filecoin focuses on data storage and thus makes it cheaper than storing it on blockchains like Bitcoin or Ethereum.

The SP uploads the result on both IPFS and Filecoin networks (via Powergate) making the result free in case of normal behaviour, and highly-available in case of the SP stops serving the result from his node. 

\paragraph{Costs}
Deploying a smart contract and invoking functions on the Ethereum blockchain costs gas. The cost of gas is proportional to the amount of computation required to execute a transaction. 

Deploying a smart contract consumed 1456577 gas.
Proof of delivery consumed 129649 gas.
Proof of provision consumed 149130 gas.

Although our experiment was conducted on testnet, the amount of gas is the same on mainnet.

The cost of gas is denominated in $\mathrm{ETH}$. The price of gas at the time of the experiment (Jan. 03, 2023, \url{https://etherscan.io/gastracker}) was $0.000000002227 \frac{\mathrm{ETH}}{\mathrm gas}$, and the price of 1 $\mathrm{ETH}$ was $1261.97 \mathrm{USD}$.

In a result, the cost of:
\begin{itemize}
  \item Deploying the smart contract cost $0.000000002227 \frac{\mathrm{ETH}}{\mathrm gas} \cdot 1456577 \mathrm{gas} \cdot 1261.97 \frac{\mathrm{USD}}{\mathrm{ETH}} \approx 4.09 \mathrm{USD}$; 
  \item Proof of delivery cost $0.000000002227 \mathrm{\frac{ETH}{gas}} \cdot 129649 \mathrm{gas} \cdot 1261.97 \frac{\mathrm{USD}}{\mathrm{ETH}} \approx 0.29 \mathrm{USD}$; 
  \item Proof of provision cost $0.000000002227 \mathrm{\frac{ETH}{gas}} \cdot 149130 \mathrm{gas} \cdot 1261.97 \frac{\mathrm{USD}}{\mathrm{ETH}} \approx 0.33 \mathrm{USD}$.
\end{itemize}


\section{Discussion}
\label{sec:discussion}

\subsection{Justice}\label{sec:decentralised-justice}

Centralized justice is the biggest obstacle in achieving a system that complies with Web3 postulates~\cite{ethereumWhatWeb3Why2023}.

The possible directions for mitigation of such issues are to either (i) replace local justice with blockchain dispute resolution systems like those proposed in Themis~\cite{mengThemisDecentralizedEscrow2019}, Kleros~\cite{bergollaKlerosSociolegalCase2022,gudkovCrowdArbitrationBlockchain2020}, Aragon Court~\cite{aragonDecentralizedDisputeResolution}, LTO Network~\cite{ltonetworkNextGenBlockchainB2B}, and other Online Dispute Resolution platforms~\cite{allenGovernanceBlockchainDispute2019}; or (ii) make it infeasible to provide incorrect results.

The first approach is more feasible in the near future. It would require creating a large set of experts in a field that, in case of dispute, would receive all the proofs ($\mathrm{PoD}$, $\mathrm{PoP}$, payment $\mathrm{receipt}$ as well as any other proofs significant to the case) that could be queried by the experts in zero-knowledge fashion, i.e., they could ask a limited number of questions to the proofs and getting yes/no answers. The case would be fully confidential as they would not be able to query personal information. The experts would be incentivized to participate in the pool by the system of fees. They would be incentivized to vote honestly by the stake they would have to lock and reward/punishment they would get by judging correctly/incorrectly, where the correctness is determined by the quorum of votes.

The second approach is more futuristic. Suppose that the service we are undertaking is fully computable. Then, it would be possible by employing proofs of correctness of computations~\cite{ben-sassonSNARKsVerifyingProgram2013} to enforce that only correct computations (hence correct services) are accepted. It would require the complete service examination to be computable, which is hard to achieve in settings where physical materials (like blood) are examined. Concretely, the problem canes down to ``How to represent blood digitally?''. If we could represent urine, blood, saliva, or any other physical material in a binary format and let the customer take a sample, discrete it, and send it to the SP, then the whole chain of integrity could be ensured. Therefore, incorrect service provision would be infeasible, and hence, disputes would be avoided.

\subsection{Self-sovereign identities}
Our research in the field showed that some jurisdictions require SPs to associate the diagnosis with the customer's personal information. For example, in Poland, all laboratories performing medical diagnostic tests, and collecting the material (except tests for HIV) are obligated to the unambiguous identification and verification of the identity of the patient from whom the material was collected~\cite{ministerstwozdrowiaRegulationMinisterHealth2006}.

This conflicts with the main goal of our protocol, which is to prevent any identification information from being collected.

One promising solution would be to use Self-sovereign identities (SSI)~\cite{muhleSurveyEssentialComponents2018}, especially the verifiable claims. A trusted authority (e.g., a government) would issue a one-time claim for a customer that would be accepted by the SP as legitimate personal identification. The SP would associate the diagnosis result with the provided DID. The DID itself would not contain any personal information, therefore the SP would learn nothing about the customer's identity besides the random-looking identification. The DID could be deanonimised by collaboration with the government.

However, since SSI is a relatively new technology and the government has not yet adopted it, we decided to postpone the discussion of this topic to future work.

\subsection{Formal Verification}\label{sec:formal-verification}
Following the work of~\cite{birjoveanuFormalVerificationMultiparty2022}, the security analysis of our protocol could be improved by the usage of tools for the automatic formal verification, e.g., AVISPA~\cite{armandoAVISPAToolAutomated2005}. It would require specifying the protocol using High-Level Protocol Specification Language (HLPSL)~\cite{chevalierHighLevelProtocol2004}. The specification would be then translated to the AVISPA's intermediate language and verified using the AVISPA's theorem prover. The verification would be performed by the AVISPA's model checker, which would check the protocol against the security properties.

\section{Conclusions}\label{sec:conclusion}
In this work, we have focused on enabling service provision without the collection of personal data. Our aim was to enable services such as genetic tests, tests for paternity, venereal diseases, HIV, drugs, and steroids, or anonymous legal advice.

We observed that the current state of the art is not sufficient to achieve the goal. Therefore, we have proposed a protocol for service provision that simultaneously achieves anonymity, fairness, dispute resolution, TTP-lessness, and supports physical materials.

Payments are handled either in cash or anonymous cryptocurrencies. The result of the service is published on a content-addressable p2p network. The dispute can be settled by disclosing the collected proofs to justice. 

Using the definition~\ref{def:fairness} we showed that the protocol achieves fairness by the proof of delivery, payment receipt, and proof of provision published on the message board. 

Finally, we pinpointed further improvements like decentralized dispute resolution, self-sovereign identifiers, and anonymous package delivery via courier.

\appendix

\section{Proof of fairness}\label{app:proof-of-fairness}
Below we describe each step and rationale of the outcome position.
We use the notation introduced in Section~\ref{sec:fairness-model} to analyze each position in the protocol and fairness Definition~\ref{def:fairness} to show the fairness of the protocol.

\newcommand{\AgreeablePath}{Agreeable path:}
\newcommand{\DisputePath}{The \customer{} starts a dispute:}
\newcommand{\Fairness}{Fairness:}
\newcommand{\CustomerTurn}[0]{\expandafter\MakeUppercase \customer{} turn:}
\newcommand{\SPTurn}[0]{\sp{} turn:}

\newcommand{\CanFollowToOne}[2]{The #1 can follow the protocol to the non-disadvantaged position #2}
\newcommand{\CanDoNothing}[1]{The #1 can do nothing and always ends up in the non-disadvantaged position}
\newcommand{\CanDoAnything}[1]{The #1 can do anything and always ends up in the non-disadvantaged position}
\newcommand{\Pos}[4]{$\operatorname{\sigma_{#1, #2, #3} = #4}$}
\newcommand{\WinForTheSameReason}[1]{The #1 wins the dispute for the same reason}
\newcommand{\LoseForTheSameReason}[1]{The #1 loses the dispute for the same reason}
\newcommand{\ActedAbnormallyThen}[1]{The #1 acted abnormally, then:}
\newcommand{\CustomerPaidButDidntGetResult}{the customer ends up in a disadvantaged position as he paid in advance but didn't receive the result}
\newcommand{\SpReceivedThePayment}{the SP ends up in the advantaged position as she received the payment}

\newcommand{\CustomerLosesBeforePayment}{The customer loses the dispute because the SP is not obligated to do anything until the transaction is paid}
\newcommand{\CustomerLosesBeforePoP}{The customer loses the dispute because the SP still can publish the PoP within the agreed timeframe}

\newcommand{\RemainsIn}[2]{The #1 remains in the #2 position}

\subsubsection*{Step 1. \CustomerTurn{} Package delivery}\label{step-1-deliver-package}

The protocol starts when the customer correctly executes the first step of the protocol, i.e., deliver the package to the SP. 

The case where the customer does not deliver the package is not considered as it is not a part of the protocol.

\begin{itemize}
\item \AgreeablePath
  \begin{itemize}
    \item  \Pos{1}{c}{\normal}{\neutral}, the customer risked his materials but did not pay for the transaction and therefore ends up in a neutral position (see Assumption 10. in Section \ref{sec:assumptions}).
    \item \Pos{1}{s}{\normal}{\neutral}, the SP ends up in a neutral position as she did not spend any resources and the package did not bring her any value.
  \end{itemize}
\item \DisputePath
  \begin{itemize}
    \item \Pos{1}{c}{\dispute}{\minus}, \CustomerLosesBeforePayment{}.
    \item \Pos{1}{s}{\dispute}{\neutral}, \WinForTheSameReason{SP}.
  \end{itemize}
\end{itemize}

\Fairness

\begin{itemize}
  \item \CanFollowToOne{customer}{\Pos{1}{c}{n}{\neutral}}
  \item \CanDoNothing{SP}
\end{itemize}

\subsubsection*{Step 2. \SPTurn{} Proof of Delivery}\label{step-2-proof-of-delivery}

The SP publishes the PoD, then:

\begin{itemize}
  \item \AgreeablePath
    \begin{itemize}
      \item \Pos{2}{c}{\normal}{\neutral}, the customer remains in the neutral position as the PoD allows him to pay for the transaction but does not obligate him to anything.
      \item \Pos{2}{s}{\normal}{\neutral}, the SP remains in the neutral position as the package has not brought her any value and she did not spend any resources to provide the service yet.
    \end{itemize}

  \item \DisputePath
    \begin{itemize}
      \item \Pos{2}{c}{\dispute}{\minus}, \CustomerLosesBeforePayment{}.
      \item \Pos{2}{s}{\dispute}{\neutral}, \WinForTheSameReason{SP}.
    \end{itemize}
\end{itemize}

\ActedAbnormallyThen{\sp}

\begin{itemize}
\item \AgreeablePath
  \begin{itemize}
    \item \Pos{2}{c}{\abnormal}{\neutral}, the customer remains in the neutral position as he is not obligated\footnote{By not obligated we understand the situation where a party does not risk any resources by not taking the action} to agree with the incorrect $\mathrm{PoD}$.
    \item \Pos{2}{s}{\abnormal}{\neutral}, the SP remains in the neutral position as the package has not brought her any value and she did not spend any resources to provide the service yet.
  \end{itemize}
\item \DisputePath
  \begin{itemize}
    \item \Pos{2}{c}{\abdispute}{\minus}, \CustomerLosesBeforePayment{}, not even to publish correct $\mathrm{PoD}$.
    \item \Pos{2}{s}{\abdispute}{\neutral}, \WinForTheSameReason{SP}.
  \end{itemize}
\end{itemize}

\Fairness

\begin{itemize}
  \item \CanDoAnything{SP}.
  \item The customer can either wait (in case of the SP following the protocol) or abandon the transaction (in case of the SP acting abnormally). In both cases the customer ends up in non-disadvantaged position \Pos{2}{c}{\normal}{\neutral} or \Pos{2}{c}{\abnormal}{\neutral}.
\end{itemize}


\subsubsection*{Step 3. \CustomerTurn{} Get Proof of Delivery}\label{step-3-get-proof-of-delivery}

The customer got the \PoD, then:

\begin{itemize}
\item \AgreeablePath
  \begin{itemize}
    \item \Pos{3}{c}{\normal}{\neutral}, \RemainsIn{customer}{neutral}.
    \item \Pos{3}{s}{\normal}{\neutral}, \RemainsIn{SP}{neutral}.
  \end{itemize}
\item \DisputePath
  \begin{itemize}
    \item \Pos{3}{c}{\dispute}{\minus}, \CustomerLosesBeforePayment{}.
    \item \Pos{3}{s}{\dispute}{\neutral}, \WinForTheSameReason{SP}.
  \end{itemize}
\end{itemize}

\ActedAbnormallyThen{\customer}

\begin{itemize}
\item \AgreeablePath
  \begin{itemize}
    \item \Pos{3}{c}{\abnormal}{\neutral}, \RemainsIn{customer}{neutral}.
    \item \Pos{3}{s}{\abnormal}{\neutral}, \RemainsIn{SP}{neutral}.
  \end{itemize}
\item \DisputePath
  \begin{itemize}
    \item \Pos{3}{c}{\abdispute}{\minus}, \CustomerLosesBeforePayment{}.
    \item \Pos{3}{s}{\abdispute}{\neutral}, \WinForTheSameReason{SP}.
  \end{itemize}
\end{itemize}

\Fairness

\begin{itemize}
  \item \CanFollowToOne{customer}{\Pos{3}{c}{\normal}{\neutral}}.
  \item \CanDoNothing{SP}.
\end{itemize}



\subsubsection*{Step 4. \CustomerTurn{} Payment}

The customer paid the transaction, then:

\begin{itemize}
\item \AgreeablePath
  \begin{itemize}
    \item \Pos{4}{c}{\normal}{\minus}, the customer paid in advance.
    \item \Pos{4}{s}{\normal}{\plus}, the SP received the payment but has not spent his resources yet.
  \end{itemize}
\item \DisputePath
  \begin{itemize}
    \item \Pos{4}{c}{\dispute}{\minus}, \CustomerLosesBeforePoP{}.
    \item \Pos{4}{s}{\dispute}{\neutral}, \WinForTheSameReason{SP}.
  \end{itemize}
\end{itemize}

\ActedAbnormallyThen{\customer}

\begin{itemize}
\item \AgreeablePath
  \begin{itemize}
    \item \Pos{4}{c}{\abnormal}{\neutral}, the customer ends up in the neutral position as he did not spend his funds.
    \item \Pos{4}{s}{\abnormal}{\neutral}, the SP ends up in the neutral position as she neither received the payment nor spent her resources.
  \end{itemize}
\item \DisputePath
  \begin{itemize}
    \item \Pos{4}{c}{\abdispute}{\minus}, \CustomerLosesBeforePayment{}.
    \item \Pos{4}{s}{\abdispute}{\neutral}, \WinForTheSameReason{SP}.
  \end{itemize}
\end{itemize}

\Fairness

\newcommand{\CustomerRiskTemporaryDisadvantagedPosition}[1]{The customer, following the 9th assumption described in Section~\ref{sec:assumptions}, risk the temporary disadvantaged position #1 in favor of a later better position \Pos{9}{s}{\normal}{\plus}; meanwhile being able to get out of the disadvantaged position if the SP misbehave in one of the following steps.}

\begin{itemize}
  \item \CustomerRiskTemporaryDisadvantagedPosition{\Pos{4}{c}{\normal}{\minus}}.
  \item \CanDoNothing{SP}.
\end{itemize}



\subsubsection*{Step 5. \SPTurn{} Provision of service}

The \sp{} did the provision of service, then:

\begin{itemize}
\item \AgreeablePath
  \begin{itemize}
    \item \Pos{5}{c}{\normal}{\minus}, \RemainsIn{\customer}{disadvantaged} as he did not receive the result.
    \item \Pos{5}{s}{\normal}{\plus}, \RemainsIn{\sp}{advantaged} as she received the payment.
  \end{itemize}
\item \DisputePath
  \begin{itemize}
    \item \Pos{5}{c}{\dispute}{\minus}, \CustomerLosesBeforePoP{}.
    \item \Pos{5}{s}{\dispute}{\neutral}, \WinForTheSameReason{SP}.
  \end{itemize}
\end{itemize}

\ActedAbnormallyThen{\sp}

\begin{itemize}
\item \AgreeablePath
  \begin{itemize}
    \item \Pos{5}{c}{\abnormal}{\minus}, \CustomerPaidButDidntGetResult{}. 
    \item \Pos{5}{s}{\abnormal}{\plus}, \SpReceivedThePayment{}.
  \end{itemize}
\item \DisputePath
  \begin{itemize}
    \item \Pos{5}{c}{\abdispute}{\neutral} The customer wins the dispute because the SP did not provide the service within the time agreed upon in the \PoD{}, and so the SP will not be able to upload the result and publish the \PoP{} on time.
    \item \Pos{5}{s}{\abdispute}{\minus}, \LoseForTheSameReason{SP}.
  \end{itemize}
\end{itemize}

\Fairness
\newcommand{\SPCanDoBothButFollowIsSafe}[1]{The SP can follow the protocol and move to the advantaged position \Pos{#1}{s}{\normal}{\plus}, or act abnormally (not provide the service) and also move to the advantaged position \Pos{#1}{s}{\abnormal}{\plus}; however, the second option puts her at risk of terminating the protocol at \Pos{#1}{s}{\abdispute}{\minus} if the customer is rational and starts a dispute; hence, the SP should choose the first option}

\begin{itemize}
  \item \CustomerRiskTemporaryDisadvantagedPosition{\Pos{5}{c}{\normal}{\minus}}.
  \item \SPCanDoBothButFollowIsSafe{5}.
\end{itemize}

\subsubsection*{Step 6. \SPTurn{} Upload result}\label{step-6-publication-of-results}

The SP uploaded the result on time, then:

\begin{itemize}
\item \AgreeablePath
  \begin{itemize}
    \item \Pos{6}{c}{\normal}{\minus}, \RemainsIn{\customer}{disadvantaged} as he did not receive the result.
    \item \Pos{6}{s}{\normal}{\plus}, \RemainsIn{SP}{advantaged} as she received the payment.
  \end{itemize}
\item \DisputePath
  \begin{itemize}
    \item \Pos{6}{c}{\dispute}{\minus}, \CustomerLosesBeforePoP{}.
    \item \Pos{6}{s}{\dispute}{\neutral}, \WinForTheSameReason{SP}.

  \end{itemize}
\end{itemize}

\ActedAbnormallyThen{\sp}

\begin{itemize}
\item \AgreeablePath
  \begin{itemize}
    \item \Pos{6}{c}{\abnormal}{\minus}, \CustomerPaidButDidntGetResult{}.
    \item \Pos{6}{s}{\abnormal}{\plus}, \SpReceivedThePayment{}.
  \end{itemize}
\item \DisputePath
  \begin{itemize}
    \item \Pos{6}{c}{\abdispute}{\neutral}, the customer wins the dispute because the SP did not upload the service within the time agreed upon in the \PoD{}, and so the SP will not be able to publish the \PoP{} on time.
    \item \Pos{6}{s}{\abdispute}{\minus}, \LoseForTheSameReason{SP}.
  \end{itemize}
\end{itemize}

\Fairness

\begin{itemize}
  \item \CustomerRiskTemporaryDisadvantagedPosition{\Pos{6}{c}{\normal}{\minus}}.
  \item \SPCanDoBothButFollowIsSafe{6}.
\end{itemize}

\subsubsection*{Step 7. \SPTurn{} Proof of provision}\label{step-7-publication-of-proof-of-provision}

The SP published \PoP{} on time, then:

\newcommand{\CustomerLosesBecauseSPCanProveBeingCorrect}{the \customer{} loses the dispute as the \sp{} has published all evidences to prove her correct behaviour}

\begin{itemize}
  \item \AgreeablePath
    \begin{itemize}
      \item \Pos{7}{c}{\normal}{\minus}, the customer has not received the result. Therefore, he remains in a disadvantaged position. 
      \item \Pos{7}{s}{\normal}{\plus}, the SP has published all evidence to prove her correct behavior, therefore she remains in an advantaged position for the rest of the protocol.
    \end{itemize}
  \item \DisputePath
    \begin{itemize}
      \item \Pos{7}{c}{\dispute}{\minus}, \CustomerLosesBecauseSPCanProveBeingCorrect{}.
      \item \Pos{7}{s}{\dispute}{\plus}, \WinForTheSameReason{SP}.
    \end{itemize}
\end{itemize}

\ActedAbnormallyThen{\sp}

\begin{itemize}
\item \AgreeablePath
  \begin{itemize}
    \item \Pos{7}{c}{\abnormal}{\minus}, \CustomerPaidButDidntGetResult{}.
    \item \Pos{7}{s}{\abnormal}{\plus}, \SpReceivedThePayment{}.
  \end{itemize}
\item \DisputePath
  \begin{itemize}
    \item \Pos{7}{c}{\abdispute}{\neutral}, the customer wins the dispute because the SP did not publish the correct \PoP{} on time.
    \item \Pos{7}{s}{\abdispute}{\minus}, \LoseForTheSameReason{SP}.
  \end{itemize}
\end{itemize}

\Fairness

\begin{itemize}
  \item \CustomerRiskTemporaryDisadvantagedPosition{\Pos{7}{c}{\normal}{\minus}}.
  \item \SPCanDoBothButFollowIsSafe{7}.
\end{itemize}


\subsubsection*{Step 8. \CustomerTurn{} Get Proof of Provision}\label{step-8-pull-proof-of-provision}

The customer got the \PoP{}, then:

\begin{itemize}
\item \AgreeablePath
  \begin{itemize}
    \item \Pos{8}{c}{\normal}{\minus}, the customer gets the \cid{}, but not the result yet.
    \item \Pos{8}{s}{\normal}{\plus}, \RemainsIn{\sp}{advantaged}.
  \end{itemize}
\item \DisputePath

  \begin{itemize}
    \item \Pos{8}{c}{\dispute}{\minus}, \CustomerLosesBecauseSPCanProveBeingCorrect{}.
    \item \Pos{8}{s}{\dispute}{\plus}, \WinForTheSameReason{SP}.
  \end{itemize}
\end{itemize}

\ActedAbnormallyThen{\customer}

\begin{itemize}
\item \AgreeablePath
  \begin{itemize}
    \item \Pos{8}{c}{\abnormal}{\minus}, the customer paid for the transaction but does not have the access to $cid$, and hence, can not get the result from the storage network.
    \item \Pos{8}{s}{\abnormal}{\plus}, \SpReceivedThePayment{}.
  \end{itemize}
\item \DisputePath
  \begin{itemize}
    \item \Pos{8}{c}{\abdispute}{\minus}, \CustomerLosesBecauseSPCanProveBeingCorrect{}
    \item \Pos{8}{s}{\abdispute}{\neutral}, \WinForTheSameReason{SP}.
  \end{itemize}
\end{itemize}


\Fairness

\begin{itemize}
  \item \CustomerRiskTemporaryDisadvantagedPosition{\Pos{8}{c}{\normal}{\minus}}.
  \item \CanDoNothing{} \Pos{8}{s}{\normal}{\plus} or \Pos{8}{s}{\abnormal}{\plus}.
\end{itemize}

\subsubsection*{Step 9. \CustomerTurn{} Download result}\label{step-9-retrieval-of-results}

The customer downloaded the result, then:

\begin{itemize}
\item \AgreeablePath
  \begin{itemize}
    \item \Pos{9}{c}{\normal}{\plus}, the customer got the result, therefore he terminate the protocol in an advantaged position.
    \item \Pos{9}{s}{\normal}{\plus}, \RemainsIn{\sp}{advantaged}.
  \end{itemize}
\item \DisputePath
  \begin{itemize}
    \item \Pos{9}{c}{\dispute}{\minus}, \CustomerLosesBecauseSPCanProveBeingCorrect{}.
    \item \Pos{9}{s}{\dispute}{\plus}, \WinForTheSameReason{SP}.
  \end{itemize}
\end{itemize}

\ActedAbnormallyThen{\customer}


\begin{itemize}
\item \AgreeablePath
  \begin{itemize}
    \item \Pos{9}{c}{\abnormal}{\minus}, the customer ends up in a disadvantaged position, as he ends up with the incorrect result.
    \item \Pos{9}{s}{\abnormal}{\plus}, the SP ends up in the advantaged position as he received the payment but did not spend his resources.
  \end{itemize}
\item \DisputePath
  \begin{itemize}
    \item \Pos{9}{c}{\abdispute}{\neutral}, the customer wins the case and ends up in the neutral position.
    \item \Pos{9}{s}{\abdispute}{\minus}, the SP loses the case and ends up in the disadvantaged position.
  \end{itemize}
\end{itemize}

\Fairness

\begin{itemize}
  \item \CanFollowToOne{\customer}{\Pos{9}{c}{\normal}{\plus}}.
  \item \CanDoNothing{} \Pos{9}{s}{\normal}{\plus} or \Pos{9}{s}{\abnormal}{\plus}. 
\end{itemize}

\bibliographystyle{spbasic}
\bibliography{bibliography}

\end{document}