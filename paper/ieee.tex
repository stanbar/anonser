\documentclass{ieeeaccess}
\usepackage{cite}

\usepackage{amsmath,amssymb,amsfonts}
\usepackage{algorithmic}
\usepackage{graphicx}
\usepackage{textcomp}
\usepackage{float}

\usepackage{caption}
\usepackage{url}
\usepackage{listings}
\usepackage{pdfpages}
\usepackage[english]{babel}
\usepackage{xcolor,colortbl}
\usepackage{scalerel}
\newtheorem{theorem}{Theorem}
\newtheorem{definition}{Definition}

\newcommand{\customer}{customer}
\renewcommand{\sp}{SP}

\newcommand{\normal}{n}
\newcommand{\dispute}{d}
\newcommand{\abnormal}{\overline{n}}
\newcommand{\abdispute}{\overline{d}}

\newcommand{\PoD}{$\mathrm{PoD}$}
\newcommand{\PoP}{$\mathrm{PoP}$}
\newcommand{\cid}{$\mathrm{cid}$}

\newcommand{\floor}[1]{\left\lfloor #1 \right\rfloor}
\newcommand{\ceil}[1]{\left\lceil #1 \right\rceil}

\newcommand{\plus}{+}
\newcommand{\minus}{-}
\newcommand\neutral[1][.75]{\mathbin{\ThisStyle{\vcenter{\hbox{%
  \scalebox{#1}{$\SavedStyle\bullet$}}}}}%
}

\newcommand\TITLE{Anonymous provision of privacy-sensitive services using distributed networks}
% \newcommand\TITLE{Privacy-preserving physical delivery service provisioning using distributed networks}
\newcommand\STANLONG{Stanis\l{}aw Bara{\'n}ski}
\newcommand\JULLONG{Julian Szyma{\'n}ski}
\newcommand\STAN{S. Bara{\'n}ski}
\newcommand\JUL{J. Szyma{\'n}ski}

\newcommand\ABSTRACT{%
Service providers like lawyers, laboratories, auditors, or banks, to provide services, require customers to submit data often associated with their personal information.
This information puts suppliers at risk of breaking GDPR, CPAA, or other privacy regulations, and puts customers at risk of privacy violations.

However, most service providers use personal information merely for logistic operations like payments and communication and could provide services anonymously if other means of logistics exist.
We propose an anonymous protocol for coordinating service provision using blockchain for both anonymous payments and proofs of existence and content-addressable network for results delivery.
% Less important
Service providers may use our protocol to provide services without collecting personal information. Enabling provision of services that were previously limited by a high level of trust.
% Even less important
Possible applications for this protocol include anonymous genetic tests, tests for paternity, venereal diseases, HIV, drugs, and steroids, or anonymous legal advice.

Compared to other works, our protocol achieves stronger anonimity by accepting payments in either cash or privacy-preserving blockchains. It supports physical materials, dispute resolution, and does not require customers to submit any transaction to the message board, indicating higher practicality.

We provide the fairness analysis and an implementation of the protocol using Ethereum as a message board, Monero as a privacy-preserving blockchain, and Powergate (IPFS and Filecoin) as a content-addressable network.
We discuss further extensions to an anonymous delivery system, self-sovereign identifiers, and a decentralized conflict resolution system.
%JS tu należałoby dodac kilka słów na temat rozwiazania i jak zostało ono zwalidowane czemu jest lepsze od innych
}

\newcommand\ACKNOWLEDGEMENTS{
The work has been supported partially by the founds of Department of Computer Architecture Faculty of Electronics, Telecommunications and Informatics, Gdańsk University of Technology.
}

\newcommand\LASTVISITED{(last visited Jan. 31, 2023)}

\newcommand{\plus}{+}
\newcommand{\minus}{-}
\newcommand\neutral[1][.75]{\mathbin{\ThisStyle{\vcenter{\hbox{%
  \scalebox{#1}{$\SavedStyle\bullet$}}}}}%
}

\newcommand{\customer}{customer}
\renewcommand{\sp}{SP}

\newcommand{\normal}{n}
\newcommand{\dispute}{d}
\newcommand{\abnormal}{\overline{n}}
\newcommand{\abdispute}{\overline{d}}

\newcommand{\PoD}{$\mathrm{PoD}$}
\newcommand{\PoP}{$\mathrm{PoP}$}
\newcommand{\cid}{$\mathrm{cid}$}

\newcommand{\floor}[1]{\left\lfloor #1 \right\rfloor}
\newcommand{\ceil}[1]{\left\lceil #1 \right\rceil}

\def\BibTeX{{\rm B\kern-.05em{\sc i\kern-.025em b}\kern-.08em
    T\kern-.1667em\lower.7ex\hbox{E}\kern-.125emX}}

\begin{document}

\history{Date of publication xxxx 00, 0000, date of current version xxxx 00, 0000.}
\doi{10.1109/ACCESS.2022.0122113}
\title{\TITLE}

\author{\uppercase{\STANLONG} \authorrefmark{1},
\uppercase{\JULLONG{}} \authorrefmark{1}} 
 

\address[1]{Department of Electronic, Telecommunication and Informatics, Gdansk University of Technology, Narutowicza 11/12 Gdansk Poland (e-mail: stanislaw.baranski@pg.edu.pl, julian.szymanski@eti.pg.edu.pl}

 

\tfootnote{The work has been supported partially by the founds of Department of Computer Architecture Faculty of Electronics, Telecommunications and Informatics, Gdańsk University of Technology.}

\markboth{\STAN{}, \JUL{} : \TITLE}
{\STAN{}, \JUL{} : \TITLE}

\corresp{Corresponding author: Stanislaw Baranski (e-mail: stanislaw.baranski@pg.edu.pl).}

\begin{abstract}
  \ABSTRACT{}
\end{abstract}

\newcommand{\sep}{,}
\begin{keywords}
Anonymity\sep{} blockchain\sep{} fair-exchange\sep{} physical delivery\sep{} privacy\sep{} services
\end{keywords}

\titlepgskip=-21pt

\maketitle

\def\JOURNALIEEE{IEEE}
\def\JOURNALELS{ELS}

\def\JOURNAL{\JOURNALIEEE}

\input{content}

\EOD


\bibliographystyle{IEEEtran}

% Loading bibliography database
\bibliography{bibliography}


\begin{IEEEbiography}[{\includegraphics[width=1in,height=1.25in,clip,keepaspectratio]{stanislaw.jpg}}]{Stanis\L{}aw Bara{\'n}ski} received BEng in 2019 and MSc in 2020 in informatics at Gdańsk University of Technology. 
Currently, he is a PhD student at the Department of Computer Architecture, Faculty of Electronics, Telecommunications and Informatics, Gdansk University of Technology.

His research interests include blockchains, applied cryptography, and secure computation; particularly the issue of blockchain-based internet voting, but also other areas that currently depend on trusted third parties.

He likes to validate his ideas in practice, developing them up to the commercialization stage. He is the author of multiple software products that have reached success in the market. His blockchain-based internet voting project has been funded by the Stellar Community Fund for the most useful Stellar application.
\end{IEEEbiography}

\begin{IEEEbiography}[{\includegraphics[width=1in,height=1.25in,clip,keepaspectratio]{julian.jpg}}]{JULIAN SZYMA{\'N}SKI}  received the PhD degree in 2009 and DSc in 2020 at Gdańsk University of Technology.
He is currently with the Department of Computer
Architecture, Faculty of Electronics, Telecommunications and Informatics, Gdansk University of Technology. He addresses the problems of knowledge representation, methods of lexical knowledge acquisition, and linguistic data utilization.
His research results find applications in web search engines and systems of automated text categorization. He manages research projects for processing the information in Wikipedia, where the main goal is to extract and structuralize textual knowledge and construct interfaces capable of communicating in natural language. Besides his research in the information retrieval domain, he was involved in projects related to cognitive science. His mainstream of this research is building semantic memory models that improve natural language processing. He also worked on analyzing human emotions, by mining EEG signals. His results are implemented in the form of brain-machine interfaces for disabled people. He also works on analyzing data within the IoT domain and its secure storage on the blockchain. 
He is involved in the project, where acquisition of data from sensors placed in beehives  allows one to predict apiary development and detect abnormalities. 

He was a committee member and reviewer in  many international
conferences as well as served as guest editor preparing special issues of prestigious journals. 
\end{IEEEbiography}

\end{document}