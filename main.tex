\documentclass{article}
\usepackage[english]{babel}
\usepackage[letterpaper,top=2cm,bottom=2cm,left=3cm,right=3cm,marginparwidth=1.75cm]{geometry}
\usepackage{amsmath}
\usepackage{graphicx}
\usepackage[colorlinks=true, allcolors=blue]{hyperref}

\title{Anonymous provision of services via blockchain}
% Anonymous and fair provision of physical services via blockchain
\author{Stanisław Barański}
\date{December 2021}

\providecommand{\keywords}[1]{\textbf{Keywords:} #1}


\begin{document}
\maketitle
\keywords{data privacy, privacy preserving, blockchain, fair-exchange}



\section{Abstract}
Service providers like: lawyers, laboratories, auditors, or banks, to provide services, require customers to submit data that is often associated with their personal information. This situation expose customers to privacy risk. However, most of the service providers use the personal information merely for logistic operations like payments and communication, and could provide services anonymously if other means of logistics would exists.

We propose a framework for coordingating local anonymous service provision. Our proposal uses blockchain for both anonymous payments and proof-of-existence, and content addressable network (IPFS) for results delivery.

In case of dispute, either due to exceeding deadlines or providing incorrect results, the customer can disclose the whole interaction and proof to the justice (police or court) the misbehaviour of the SP.

We analyse the protocol in the context of fairness, privacy, and anonymity.

We propose three applications for our protocol: (i) anonymous steroid examination, (ii) anonymous legal advice, and (iii) anonymous financial trustworthiness evaluation.

We contribute to the field of fair-exchanges by proposing a simple and practical framework that does not rely on centralized trusted-third party, rather on decentralized blockchain and decentralized content addressable networking. Also, we support physical materials delivery, which was rarely targeted by other researchers.

\section{Introduction}
Service providers (SP) like: lawyers, laboratories, auditors, or banks, to provide services, require data that is often associated with user identity.

Providing personal information exposes users to privacy risk, i.e. potential loss of control over personal information ~\cite{smith2011information}. 

Such user information can be then used deliberately or unintentionally (e.g, by theft) for: insider disclosure, unauthorized access, or commercial gains. For example, by reselling it to marketers, financial institutions, other businesses, government agencies, or even cybercriminals. Which in turn can lead to profiled advertisements, or criminal activities like identity theft or illegal tracking and surveillance~\cite{smith2011information}.

Important individuals like top government, influencers, or celebrities are especially vulnerable to this kind of attacks as exposure of their health records, purchase habits, or legal documents can threaten their reputation, position, or can be used for blackmailing.

The guarantee of the privacy of the data is often based merely on a trust assumptions and security of IT system. However, most of the SP don't need the personal information for any other reasons than payment, or communication, which should be considered secondary compared to primary matter that is service provision.

It would be desirable if the user could keep its identity private, while service provider still provide its services. This could lead to reduced trust that users have to put on SP and less responsibility borne by SP.

\subsection{Contribution}
We contribute to the field of fair-exchanges by proposing a simple and practical framework that does not rely on centralized trusted-third party, rather on decentralized blockchain and distributed content addressable networking. 
Also, we support physical materials delivery, which was rarely targeted by other researchers.

\section{State of the art}
\subsection{Anonymity and Fair-Exchange in e-Commerce Protocol for Physical Products Delivery}

Authors of ~\cite{birjoveanu2015anonymity} proposed a fair-exchange of physical product and electronic payment protocol that guarantee both customer and merchant anonymity. They achieve it by introducing online trusted third party that validate coins and provide fair exchange guarantees.

The anonymity is ensured by assuming the existence of source cabinet (SC) and delivery cabinet (DC), and a trusted delivery agent who take product from a source cabinet and provide it to a destination cabinet.
Both cabinets provides access to the product by passwords to conceal identity of the customer and the merchant.
To avoid identity disclosure of customer and merchant, the access to the product is provided by passwords.

Both cabinets are equipment with cameras. Also there should exist the trusted delivery agent (DA) who will deliver items from source cabinet do delivery cabinet.

The protocol works as follows:
\begingroup
\renewcommand{\labelenumii}{\arabic{enumii}.}
\begin{enumerate}
\item Both parties agree on the purchase details.
\item The customer buys a digital coin from his bank and validates it with the TTP.
\item The customer sends to the merchant the purchase order and the digital signature made by TTP on the encrypted coin.
\item The merchant post the product to the source cabinet.
\item Delivery agent collect the product from source cabinet and post it to delivery cabinet.
\item Customer collects the product from delivery cabinet using password.
\item The customer checks if the product is actually the ordered one.
\begin{itemize}
\item[-] if yes
\begin{enumerate}
\setcounter{enumii}{7}
\item the customer sends the acknowledge to delivery cabinet and decryption key to the merchant.
\item The merchant redeem the coins from the bank.
\end{enumerate}
\item[-] if no
\begin{enumerate}
\setcounter{enumii}{7}
\item the delivery cabinet is equipment with video camera that records the moment when the customer open the package and let the customer to signal the invalidity of the product, and so, sending the encrypted recording to the TTP.
\item the dispute is settled via TTP, optionally letting each party to reveal it's identity.
\end{enumerate}
\end{itemize}
\end{enumerate}
\endgroup

Yet, the fair-exchange properties are achieved by strong assumptions. Namely,
\begin{itemize}
\item the TTP does not misbehave or collude with any party.
\item The customer's and merchant's banks enable confidental transactions and both share commit-buffer where the value is locked until the transaction is finished.
\item All banks mantain a global list of coin's serial numbers to prevent double-spending problem.
\item The source cabinet and delivery cabinet exists, and in both the product access is protected by password.
\item Delivery cabinet is equppted with a video camera that records the moment when the customer opens the package, and provide a way to submit the video to TTP in case of a dispute.
\item Delivery agent exist and is trusted.
\item Communication channels between parties provides anonymity.
\end{itemize}

\subsection{Lelantos - A blockchain-based anonymous physical delivery system}

Lelanros~\cite{altawy2017lelantos} is a blockchain-based anonymous physical delivery system. The anonymity is achieved by employing ToR's onion routing to connect physical delivery providers. The whole route from merchant to customer is split into multiple steps, and each step is undertaken by different randomly selected delivery provider. As long as the develiery providers does no collude, mechant can not learn both identity and destination address of the customer.


Onchain smart-contract is used to coordinate the whole process and mediate communication between customer and delivery providers.


However, the system achieve only pseudo-anonymity or 'weak anonymity'. 

\subsection{Themis - Towards Decentralized Escrow of Cryptocurrencies without Trusted Third Parties}

Themis~\cite{meng2019themis} is a fair-exchange protocol, which uses blockchain instead of TTP. It provide escrow services for secure exchanging between cryptocurrencies and digital goods.

Authors highlight that a widely-used atomic exchange mechanism HTLC (Hashed Time-lock Contract) is vulnerable to denial of service (DoS) attacks. While Multi-Signature Schemes depends on the Trusted Third Party (TTP) making them vulnerable to collusion attacks. Such TTP can collude with one party and take the locked funds. It can also refuse settling disputes and lock the hosted funds. 

Themis is a decentralized system that provides an escrow and dispute resolution. It claims to satisfy following requirements:
\begin{description}
    \item[Fairness] Either both seller and buyer obtain all the goods, or they both obtain nothing.
    \item[Security] The funds are locked until the transaction is completed.
    \item[Passivity] If no dispute arise, the third-party does not need to involve in the transaction.
    \item[Correctness] Transactions and disputes are executed according to a priori agreed protocol rules.
    \item[Dependability] There is no single point of failure vulnerable to DoS attack.
    \item[Privacy] If the dispute does not arise, the third-party does not know if the transaction completes. If the dispute arises, only the related parties are aware of it.    
\end{description}

Two parties, say Alice and Bob, can create an escrow by generating 2-of-2 (not 2-of-3 as it is other TTP-based systems) threshold account using Thresh-Key-Gen protocol. Alice and Bob generate secret keys $x_A$ and $x_B$ accordingly, the shared account public key becomes $y = g^{x_A+x_B}$; only the access to both Alice's and Bob's secrets grants access to the shared account.

Then, both Alice and Bob, takes their keys, and split them into $n=2t+1$ secret shares using Shamir Secret Shamir protocol, where $n$ is the number of mediators that participate in the decentralized network, and $t+1$ becomes the threshold of sufficient number of mediators that can reconstruct the secret key. 

Next, both Alice and Bob, take their key shares, and encrypt each $i$-th key share using public key of $i$-th mediator. As a result Alice's secret $x_A$ becomes ${c^A_1, c^A_2,...,c^A_n}$, and Bob's secret $x_B$ becomes ${c^B_1, c^B_2,...,c^B_n}$, where $c_i = E_{M_i}(P_i)$ is the encrypted $i$-th key share using $i$-th mediator public key $M_i$. 

Next, both Alice and Bob exchange the sets of encrypted key shares with each other. And send funds to the escrow account. 

The escrow is secure as long as $t+1$ of mediators does not collude, which would let them recreate both $x_A$ and $x_B$. Also, to ensure that parties are exchanging real key shares, they also send a witnesses generated with Feldman VSS scheme and zero-knowledge proofs to guarantee the consistency between witnesses and key shares.

In case of dispute, the decentralized network of mediators settle the conflict and grand one winning party, the other party secret key, letting him to withdraw the funds.

The honesty of mediators is achieved by the monetary incentivization and reputation system.

\subsection{Background}
\subsection{Dispute resolution system}

[[Themis - Towards Decentralized Escrow of Cryptocurrencies without Trusted Third Parties]] proposed a semi-autonomous decentralised dispute resolution system, which comply with the web3 postulates of decentralised web. Themis resolve disputes by set of voluntary anonymous users participating in a voting and deciding which party misbehaved. 

Such dispute resolution system might work for straightforward money exchanges. However it's hard to realise such decentralised evaluation of quality of heath or legal services when expertise and privacy concerns are taken into an account. Therefore, in this protocol we take the more practical approach and resolve disputes using local justice system. 

Possible directions toward semi-autonomous decentralised resolution system are discussed in [[anonser/Future Work]].

Unless such expertise service evaluation is possible the evaluation have to be in justice. 



\subsection{There is really no need to create an escrow}
One could assume that the escrow mechanism would make the whole protocol more fair, because the SP would not receive the money until it doesn't provide the services. But, in scenario without TTP its the SP who decide if service is provided or not. In other words, SP can finish the protocol at any time, sending some arbitrary data and claim that he finished providing the services. We could make the customer decide if the service is provided or not, but then the customer could receive the services and lie that he didn't. And since the customer is anonymous, there would be no means to open the dispute.


\subsection{Proof to justice}
In case of dispute customer can proof to the justice (police, court, etc.) evidences that convinces justice about the honesty of customer and misbehaviour of SP. Abstracting from the services the SP is providing. Customer should be able to proof that: (I) he delivered complete $pkg$ and POD proofs that; (ii) he paid for the transaction $tx$ and payment disclosure proofs that.
In case the customer is laying, and the SP has indeed delivered the results in time, he can proof: that the results had been published on time and Proof-of-Provision proofs that;


# Proof-of-Delivery

# Payment disclosure

Private cryptocurrencies can hide

# Proof-of-Provision
Blockchain is global block that time stamps everything that gets to the block. Therefore by publishing results on a blockchain, everyone is convinced that the service have been indeed provisioned on time, because the results have been available since the publication of that block. 


\subsection{Pseudo-anonymous vs anonymous blockchain}
Privacy is a concept used in almost all kinds of social sciences like philosophy, psychology, sociology, and legal. The multidisciplinary nature leads to ambigous definitions. Therefore we do not attempt to pick any definion of privacy, rather build on more concrete concept that is anonymity.[[Information Privacy Research]]

Anonymity is a the ability to hide one identity from others. More precisely, the unability to correlate actions performed within the sytem to the identity of the user. Or we can say, that the system guarantee anonimity if for all observers the actions are equally likely to be associated with any user of the system.
However, anonimity is a spectrum rather than dichotomous classification. One method of quantifying level of anonymity is proposed in [[k-anonymity]]. It measure the user's anonymity by the number of other users from whom the user can not be distinguished. Concreatly, the user is *k*-anonymous if his actions are equaly likely associated with *k*-1 other users. The larger *k*, the higher anonimity. The cryptocurrencies that natively support anonymous transactions are [[Monero]] (via [[Confidential Transactions]]) and [[ZCash]] (via [[zkSNARK]]s). There are also techniques that can be deployed on non-anonymous blockchains, they are so-called mixers [[Tornado Cash]], [[CoinJoin]], cite more..., that literaly gather together "anonimimity set" of users that collude together in order to launder the transactions, in such a way that to the outside observer the likelyhood of the sender of each transaction was equiprobable for any user from the anonimity set.

Some systems guarantee pseudonomity rather than anonimity. Pseudonomity, allows uses to hide their real identity behind the pseudonim. Despite the whole system is transparent and allows linking actions to one identity, the system is considered anonymous as long as the link between pseudonym and real identity is secret. This assumption is hard to satisfy in practice, as the KYC (Know You Customer) and AML (Anti Money Laundering) regulations require users to reveal their real identity on most of the cryptocurrencies exchanges. This makes the anonymity of the user dependent on the security of the IT system. Moreover, some correlations can be infered merely by the analysis of transactions [[Evaluating user privacy in bitcoin]], [[Structure and anonymity of the bitcoin transaction graph]]. Examples of pseudonomous blockchains are [[Bitcoin]] and [[Ethereum]].


\subsection{Payment for services}
Transactions between customers and SP have to be somehow pegged in order to prevent proving one payment for multiple transactions. In other words, we need some way to uniquely link particular payment with the transaction it pays for. 

Depending for the cryptocurrency the link can be created in different ways. 
- separate address - each transaction derives new address uniquely associating it with the transaction. Such addresses can be derived using BEP-32 standard. 
- memo - the payments are send to one SP account, but contain extra field named "memo" which contains a random value generated by SP and included in POD. Each payment containing such value in memo is considered to pay for the transaction in question. in case of dispute, the connection can be disclosed by revealing the value in POD and  payment's memo. 


Both approaches have some advantages and shortcomings, but our protocol can abstract them away by treating them as come unique number called "invoice". How it's implemented  depends on the payment blockchain. 

\subsection{Message Board}
Blockchain as a Message Board — proof-of-existence.

\subsection{We could use one blockchain}
Instead of using two blockchains, one for anonymous payments and one for message board, we could use one blockchain that supports both.

Ufortunately at the time of writing. Monero does not support message board functionallity. Zcash transaction disclosure is experimental feature.


\section{The framework}

\subsection{Overview of the framework}
\subsubsection{Why we need to first give the items to the merchant then pay}

If we pay first, then mechant can refuse to accept our package, and refuse to give us the Proof-of-Delivery acknowledge. In such a case, we are left with proof of payment that we can't really do much about it.

We have to first, go to merchant, give it package with sample, they give us the proof-of-delivery with some random secret, acting like a invoice.

Proof-of-Delivery is something we can use to proof to the justice that we have.

To prove the payment for that invoice, we have to send a message to the mechant with the memo value of the secret. Pegging these two values.

\subsubsection{There is a need to create proof of delivery}

Without the proof of delivery, the customer get's left with just proof of payment, but he has no proof that he indeed brought the sample and other stuff to the merchant.

Merchant has to give the customer either on blockchain or via other means that he accepted the package.

\subsection{Notation}
\begin{itemize}
    \item $\mathrm{SP}$ - Service Provider.
    \item $\mathrm{customer}$ - Customer.
    \item $\mathrm{pkg}$ - package anonymously delivered by customer to SP. pkg = (materials, )
    \item $\mathrm{materials}$ - materials used by SP to provide services, can be of any kind: urine, saliva, blood, stool, or health documents, legal documents etc.
    \item $\mathrm{results}$ - results in binary format, can be either PDF, docx or any format agreed by both parities. 
    \item $\mathrm{examination}(\mathrm{materials}) : \mathrm{results}$ - actual service provision executed on materials provided by customer and producing results
    \item $\mathrm{POD}$ - Proof-of-Delivery, a confirmation issued by SP to the customer. The POD encompasses delivery time and an address to send funds to. POD is signed by SP to guarantee non-... .  POD can be used as a evidence in case of dispute.
    \item $\mathrm{POP}$ - Proof-of-Provision, an logical proof of the existence of provision results.
    \item $\mathrm{E}_\mathrm{key}(\cdot)$, $\mathrm{D}_\mathrm{key}(\cdot)$ - symmetric encryption and decryption operators.
    \item $\mathrm{S}_\mathrm{key}(\cdot)$, $\mathrm{V}_\mathrm{key}(\cdot)$ - digital sign and verify operators.
    \item $\mathrm{sk}$, $\mathrm{pk}$ - Key pair of secret and public keys accordingly.
    \item $\mathrm{DH}(\cdot, \cdot)$ - Diffie–Hellman key exchange.
    \item $CID$ - Content Identifier, a label used to address content in IPFS.
\end{itemize}

\subsection{Prerequisites}
Blockchain:
- Anonymous payments.
- Invoice mechanism as described in [[Payment for services]].
- Transaction disclosure as described in [[Proof to justice]].

\subsection{Assumptions}
Our protocol rely on some assumptions:
1. SPs accept packages from unknown customers and give them back confirmation of acceptance (to stick with convention we call it Proof-of-Delivery).
2. SPs accept payment for services via anonymous cryptocurrencies like Monero, Grin or ZCash. 
3. Justice accepts as a evidence in dispute proof of payment of cryptocurrencies. 

\subsection{The protocol}
\subsubsection{Messages}

\begin{figure}[h!]
    \centering
    \includegraphics[width=\textwidth]{diagram.png}
    \caption{Diagram}
    \label{fig:diagram}
\end{figure}

\paragraph{Proof-of-Delivery}
$\mathrm{POD} = (issued-at, pay-invoice-time-limit, provide-results-time-limit, invoice-nonce, signature)$

where:
\begin{itemize}
    \item $issued-at$ - time at which the POD is issued
    \item $pay-invoice-time-limit$ - time limit to pay the invoice 
    \item $provide-results-time-limit$ - time limit to provide the service results
    \item $\mathrm{nonce}$ - randomly generated number uniquely identifying the transaction.
    \item $signature$ - signature proving authenticity of the SP
\end{itemize}
moreover

$issued-at < pay-invoice-time-limit < provide-results-time-limit$

\paragraph{Proof-of-Provision}
$\mathrm{POP} = (\mathrm{cid}, \mathrm{nonce}, signature)$

where:
\begin{itemize}
    \item $\mathrm{cid}$ - content identifier.
    \item $\mathrm{nonce}$ - randomly generated number uniquely identifying the transaction.
\end{itemize}

\subsubsection{Step 1: deliver package}

They protocol start once the customer delivers package $\mathrm{pkg}$ to SP.

In return, the SP issue an Proof-of-Delivery (POD) confirmation. The confirmation encompasses timestamp, invoice (see section [[Payment for services]]), and signature. 

Symbolically:
$$
\mathrm{delivery}(\mathrm{pkg}) : \mathrm{POD}
$$


\subsubsection{Step 2: pay for invoice}
Having POD, SP can not reject receiving the package. At this point the customer should pay for the invoice according to selected cryptocurrency system (see section [[Payment for services]]). 

In return, the customer receives the proof of payment that should be used in case of a dispute.

Symbolically:
$$
\mathrm{payment}(\mathrm{POD}.\mathrm{nonce}) : \mathrm{POP}
$$

\subsubsection{Step 3: provision of service}
At this point the SP can start the provision of service. If it's suspicious it may decide to wait until the invoice is paid, otherwise start immediately.

Symbolically:
$$
\mathrm{examination}(\mathrm{pkg}.\mathrm{materials}) : \mathrm{results}
$$

\subsubsection{Step 4: publication of results}
Once the service is finished, some kind of output should be created. The output should be encrypted using symmetric key derived using Diffie–Hellman key exchange between customer's public key and SP's secret key, and later decrypted using customer's secret key and SP's public key. 

The encrypted results can be then published on content network such as IPFS. As a result, the content identifier (CID) is returned.

Symbolically:
$$\mathrm{key} = \mathrm{DH}(\mathrm{pkg}.\mathrm{pk}, \mathrm{SP}.\mathrm{sk})$$
$$\mathrm{cid} = \mathrm{publish}(E_\mathrm{key}(\mathrm{results}))$$

\subsubsection{Step 5: publication of proof-of-provision}
Now, when the results are published on content addressable network such as IPFS, the SP publish $\mathrm{cid}$ obtained in previous step on a message board doing proof of provision. 

Symbolically:
$$\mathrm{publish}(\mathrm{cid})$$

\subsubsection{Step 6: observe proof-of-provision}

Since the delivery and payment was made the customer observes blockchain for a POP corresponding to his transaction. When it happen,   

\subsubsection{Step 7: fetch the results}




\section{Security analysis}
\subsection{Fairness analysis}
\subsubsection{Model}
We consider three positions:
\begin{itemize}
\item Neutral (•) position is when a party does not gain or lose anything significant. For example, beggining of the protocol.
\item Disadvantage (-) position if a party has put any effort or paid money without being able to reclaim it. In other words it has to **risk** its resources.
\item Advantage(+) position is when a party would benefit if the transaction would halt at that point. For example, received payment in advance.
\end{itemize}

$$\sigma = \{-, •, +\}$$

We call one position *better* than the other if it's more preferable or more advantageous for the party. Then, we say that neutral position (•) *is better* than disadvantage position (-), and advantage position (+) is better than neutral position (•). The relation is transitive so advantage position (+) is better than disadvantage position (-). Formally: 
$$
\mathrm{is\ better} = \{(-,•), (•,+), (-,+)\}
$$

There are many actions that each party can take, but we group them together into two categories: 
\begin{itemize}
    \item Following the protocol — taking the action prescribed by the protocol.
    \item Not following the protocol — everything that deviates from designed steps of the protocol. For example, sending arbitrary message, skipping some step, repeating step, timing out. 
\end{itemize}

Moreover, at any step of the protocol the customer can react with two reactions: 
\begin{itemize}
    \item Agree — the customer agree with the action and therefore does not start a dispute.
    \item Start a dispute — the customer does not agree with the action and therefore start a dispute.
\end{itemize}

Hence, in our analysis we have to consider four different outcomes for each party of the protocol:
- $\sigma_\mathrm{n}$ — after following the protocol within the timeframe.
- $\sigma_\mathrm{d}$ — after a dispute where the other party hasn't broken the protocol.
- $\sigma_\overline{n}$ — after not starting a dispute despite the other party has broken the protocol.
- $\sigma_\overline{d}$ — after a dispute where the other party has broken the protocol.

After starting a dispute, the protocol terminate. After timing out there is no going back on the normal path.

In our framework, the only party that can start a dispute is the customer. It's because the customer is anonymous, and so the SP does not have any personal information to start a dispute against. 

To mitigate this issue, the protocol is designed so that the SP who follows the protocol is always in advantage position, therefore he has no incentivisations to start a dispute. 

We say that the protocol achieve fairness if:

$$\forall_{party \in parties}\forall_{\mathrm{step} \in \mathrm{steps}} \mathrm{can\ move\ the\ protocol\ forward\ to\ the\ non-disadvantage\ position}$$


\subsubsection{Assumptions}
For the analysis purpose we state assumptions listed below. We consider them being rational and realistic.

Assumptions:
\begin{itemize}
    
\item Both parties start from neutral position (•).
\item Both parties, by completing the transaction, end up in advantage positions (+). The transaction produces win-win scenario. In other words, they have intrinsic motivation to pursue in the transaction.
\item We assume the steps of the protocol to be atomic, there are no intermediate steps.
\item Repeating the first step starts new transaction. Repeating any other step is considered abnormal and gets ignored. For example paying for the invoice twice does not cause any effect on the curse of protocol.
\item The protocol can go only forward, there is no way of revert any action.
\item Losing the dispute leads to punishment that is greater than any reward therefore always leads to disadvantage position. Hence, the rational customer won't start a dispute that he will lose.
\item Winning the dispute leads to neutral position (•)
\item Both the customer and the SP are rational (selfish). They always prefer go from worst position to better position, but also risk temporary worst position in favour for later better position iff it's assured that they won't stuck in worst position. 
\end{itemize}

Motivation:
At least one party should be motivated either via reward (better possition) or punishment (worst position) to move the protocol forward.

The protocol is fair, if at any point of the protocol, each party is either in Disputable or Advantage position. Moreover, at least one party has the motivation and means to move the protocol forward. 

\begin{figure}[h!]
    \centering
    \includegraphics[width=\textwidth]{table-of-positions.png}
    \caption{positions}
    \label{fig:my_label}
\end{figure}

\subsubsection{Steps}
We assume the steps of the protocol to be atomic, there are no intermediate steps, nor the party can revert any step. In other words, it can only move forward.

\subsubsection{Step 1: deliver package}
Turn: Customer 

Customer has delivered the package:
- Agreeable path:
	- $\sigma_{1, c, \mathrm{n}} = •$ - the customer has risked its materials but has not paid anything, so we assume he is in neutral position (•). One would argue that the customer has put effort of delivering the materials to the merchant, or that the sample is valuable. We assume that the sample without personal information is useless and the effort to deliver the package is negligible.
	- $\sigma_{1, s, \mathrm{n}} = •$ - the SP is in neutral position as he didn't put any effort to rece the package, also the package hasn't any value to the SP (•).
- Starting a dispute:
	- $\sigma_{1, c, \mathrm{d}} = -$ - the customer lose the case as the SP has still opportunity to publish proof-of-provision within timeframe (-).
	- $\sigma_{1, s, \mathrm{d}} = •$ - the SP win the case for the same reason (•).

Fairness:
- The customer can follow the protocol and move to the non-disadvantage position $\sigma_{1, c, \mathrm{n}} = •$.
- The SP can do nothing and end up in non-disadvantage position $\sigma_{1, s, \mathrm{n}} = •$ or $\sigma_{1, s, \mathrm{d}} = •$.

\subsubsection{Step 2: pay for invoice}
Turn: Customer

Customer has paid the invoice:
\begin{itemize}
    \item Following protocol:
    \begin{itemize}
	    \item $\sigma_{2, c, \mathrm{n}} = -$ the customer has spend his funds but hasn't received the results. Therefore, he moves to the disadvantage position (-).
	    \item  $\sigma_{2, s, \mathrm{n}} = +$ the SP has received the payment but hasn't spend his resources yet. Therefore, he moves to advantage position (+).
    \end{itemize}
    \item Starting a dispute:
    \begin{itemize}
	\item $\sigma_{2, c, \mathrm{d}} = -$ the customer lose the case as the SP has still opportunity to publish proof-of-provision before the deadline (-)
	\item $\sigma_{2, s, \mathrm{d}} = +$ the SP win the case for the same reason (+)
	\end{itemize}
\end{itemize}

Customer has timeouted $time-to-issue-$, then:
- Following protocol:
	- $\sigma_{2, c, \overline{\mathrm{n}}} = •$ - the customer ends up in neutral position, since he didn't spend his funds (•)
	- $\sigma_{2, s, \overline{\mathrm{n}}} = •$ - The SP ends up in neutral position as he didn't receive the payment, nor spent his resources (•)
- Starting a dispute:
	- $\sigma_{2, c, \overline{\mathrm{d}}} = -$ - the customer loses the case as he can not prove the payment before the time limit. Consequently, ends up in disadvantage position (-)
	- $\sigma_{2, s, \overline{\mathrm{d}}} = •$ - the SP is not charged to the justice due to lack of proofs. Therefore, he ends up in neutral position (•)


Fairness:
\begin{itemize}
    \item The customer can move the protocol forward to the step 3, where if the SP follow the protocol then the customer can push the protocol forward up to the last termination step $\sigma_{7, c, \mathrm{n}} = +$, otherwise if at any step the SP does not follow the protocol the customer can start a dispute which terminates the protocol at non-disadvantage position, i.e., $\sigma_{3, c, \overline{\mathrm{d}}} = •$, $\sigma_{4, c, \overline{\mathrm{d}}} = •$, $\sigma_{5, c, \overline{\mathrm{d}}} = •$.
    \item The SP can do nothing and end up in non-disadvantage position $\sigma_{2, s, \mathrm{n}} = +$, $\sigma_{2, s, \mathrm{d}} = +$, $\sigma_{2, s, \overline{\mathrm{n}}} = •$, $\sigma_{2, s, \overline{\mathrm{d}}} = •$.
\end{itemize}

\subsubsection{Step 3: provision of service}

SP has done the provision of service:
- Following protocol:
	- The customer hasn't received the results. Therefore, he remains in disadvantage position (-).
	- The SP has fullfiled the contract on time (+).
- Starting a dispute:
	- The customer lose the case as the SP has still opportunity to publish proof-of-provision within timeframe (-)
	- The SP win the case for the same reason (+)

SP has timeouted, then:
- Following protocol:
	- The customer ends up in disadvantage position (-)
	- The SP ends up in advantage position as he received the payment (+)
- Starting a dispute:
	- The customer wins the case and end up in neutral position (•)
	- The SP loses the case and end up in disadvantage position (-)


Motivation:
The SP has the motivation to move the protocol forward, as he don't want to lose the case and get punished.

\subsubsection{Step 4: publication of results}

SP has published results on time:
- Following protocol:
	- The customer hasn't received the results. Therefore, he remains in disadvantage position (-).
	- The SP has fullfiled the contract on time (+).
- Starting a dispute:
	- The customer lose the case as the SP has still opportunity to publish proof-of-provision within timeframe (-)
	- The SP win the case for the same reason (+)

SP has timeouted, then:
- Following protocol:
	- The customer ends up in disadvantage position (-)
	- The SP ends up in advantage position as he received the payment (+)
- Starting a dispute:
	- The customer wins the case and end up in neutral position (•)
	- The SP loses the case and end up in disadvantage position (-)


Motivation:
The SP has the motivation to move the protocol forward, as he does not want to timeout and expose to losing case.

\subsubsection{Step 5: publication of proof-of-provision}

SP has published proof-of-provision on time:
- Following protocol:
	- The customer hasn't received the results. Therefore, he remains in disadvantage position (-).
	- The SP has fullfiled the contract on time (+).
- Starting a dispute:
	- The customer lose the case as the SP can prove the publication of proof-of-provision (-)
	- The SP win the case as the SP can prove the publication of proof-of-provision (+)

SP has timeouted, then:
- Following protocol:
	- The customer ends up in disadvantage position (-)
	- The SP ends up in advantage position as he received the payment (+)
- Starting a dispute:
	- The customer wins the case and end up in neutral position (•)
	- The SP loses the case and end up in disadvantage position (-)


Motivation:
The customer wants to receive the results, therefore he move the protocol forward.

\subsubsection{Step 6: retreival of proof-of-provision}

SP has published results on time, then:
- Follows the protocol:
	- The customer gets hash of the results but not the results. Therefore, he is still in disadvantage position(-).
	- The SP position hasn't changed(+).
- Starts a dispute:
	- The customer lose the case as the SP can prove the publication of proof-of-provision (-)
	- The SP win the case as the SP can prove the publication of proof-of-provision (+)

Customer has timeouted, then:
- Follows the protocol:
	- The customer ends up in disadvantage position (-)
	- The SP ends up in advantage position as he received the payment but didn't spend his resources (+)
- Starts a dispute:
	- The customer wins the case and end up in neutral position (•)
	- The SP loses the case and end up in disadvantage position (-)


Motivation:
The customer wants to receive the results, therefore he move to the protocol forward.

\subsubsection{Step 7: retrieval of results}

SP has published **correct** results on time:
- Following protocol:
	- The customer gets the correct results. Therefore, he moves to advantage position (+).
	- The SP position hasn't changed(+).
- Starting a dispute:
	- The customer lose the case as the SP can prove the publication of proof-of-provision (-)
	- The SP win the case as the SP can prove the publication of proof-of-provision (+)

SP has published **incorrect** results on time:
- Following protocol:
	- The customer end up in disadvantage position, as he ends up with incorrect results (-)
	- The SP ends up in advantage position as he received the payment but didn't spend his resources (+)
- Starting a dispute:
	- The customer wins the case and ends up in neutral position (•)
	- The SP lose the case and end up in disadvantage position (-)

\subsubsection{Examples}


\begin{figure}[h!]
    \centering
    \includegraphics[width=\textwidth]{misbehaving-path.png}
    \caption{positions}
    \label{fig:my_label}
\end{figure}

\begin{figure}[h!]
    \centering
    \includegraphics[width=\textwidth]{rational-path.png}
    \caption{rational path}
    \label{fig:my_label}
\end{figure}


Motivation:
At this point no party has the motivation to start the case, since everyone  . 


See how monero swaps analysis fairness and motivations.


\subsection{Privacy and anonymity analysis}

# Model
The data can be either:
- Public(+), 
- Private(-)

- Plaintext (...)
- Encrypted (...)

We analisis at each step of the protocol

\section{Applications}
\subsection{Anonymous clinical tests}
Bodybuilders, athletes,  weightlifters, cyclists, football and rugby players have to pass anti-doping tests before attending the competition. 

Patients willing to check steroid tests, drug tests, venereal diseases — anything that they wouldn't want to share with anyone, have to risk that the data stored in laboratory won't get to unauthorized hands like attacker or malicious worker.

Analysis of urine, saliva, stool, or blood. 

\subsection{Anonymous lawyer consultations}

People willing to take business action that may be risky or not yet fully legistlated, may be willing to receive the estimation of risks and repercussions of such action.
Such protocol would provide a mechanism for 


\subsection{Anonymous financial trustworthiness evaluation}

Evaluating financial trustworthiness is required when user want to, for example issue a credit card, take loan for a home, or rent a car. 
As defined in~\cite{lin2021efficient}, such a system consist of three parties: (i) users, (ii) credit bureaus, and (iii) creditors. Users willing to take a credit, have to provide to the credit bureaus, their financial activities like: new account creation, account balance, credit card utilization, credit inquires, and payment history. Such information are then submitted to creditors to compute the user's credit score. 

\subsection{Anonymous physical delivery e-commerce}
% TODO write about
~\cite{birjoveanu2015anonymity}

\section{Future work}

\subsection{Justice}
Justice is the one biggest obstacle in achieving system that is complied with web3 manifests. 

The possible directions for mitigation of such issue are to either (1) replace Justice with decentralised autonomous organisations (DAO) like proposed in [[Themis - Towards Decentralized Escrow of Cryptocurrencies without Trusted Third Parties]]; or (2) make it impossible to provide incorrect results.

The first approach is more feasible in the near future. It would require creating large pool of experts in a field, that in case of dispute, would receive all the proofs (POD, POP, POP, as well as any other proofs significant to the case) that could be queried by the experts in zero-knowledge fashion, i.e., they could ask limited number of questions to the proofs and getting yes/no answers. Keeping the case fully confidential. The experts would be incentivised to participate in the pool by the system of fees. Their honest behaviour would be incentivised by the stake they would have to lock, and punishment they would get by judging incorrectly, where the "correctly" is determined by the quorum of votes.

The second approach is more philosophical and visionary. Suppose that the service we are undertaking is fully computable. Then, it would be possible by employing proofs of correctness of computations (cite snarks) to enforce that only correct computations (hence correct services) are accepted. But to do so the whole service examination would have to be computable, which is hard to achieve in settings where physical materials (like urine) are examined. Concretely, the problem canes down to "How to digitally represent urine". If we could represent urine, blood, saliva, or any other physical material in binary format, and let the customer to take a sample, discrete it and send it to the SP by himself. Then the whole chain of integrity could be ensured. Therefore, wrong service provision would be impossible, or — using cryptographic nomenclature — infeasible. 

\subsection{Paying with cash}
Without lose of generality the payment can be made with cash. 
The customer can exchange the cash with SP together with delivering the package pkg. The POD would contain an information that the transaction has been paid with cash; therefore, [[The protocol#Phase 2 pay for invoice]] could be ommited. 


%%# Verifable documents

Verifiable Document Redacting ~\cite{chabanne2017verifiable} is a protocol for erasing confidential information from authenticated images in verifiable manner. In other words, the protocol allows verifying that only the fields that cover personal information has been blacked out from the authenticated image. This allows users to provide required images without exposing their personal information, and SPs to authenticate the modified images, such that they are sure that only the stated fields have been blacked out and nothing else was touched. 

If such Verifiable Redacted Documents were accepted by other service providers like bank or tax office, national register of the judiciary, then the SP could use the documents provided by the customer to provide wider range of services, without knowing identity of the customer.%%


\subsection{Conclusions}
In this paper, we have proposed a privacy- and anonymity- preserving blockchain-based framework for local services provision. The framework can be employed by any local service provider to provide services without collecting any personal information. The payment is handled by anonymous cryptocurrencies, and the results are published on content based p2p network. The dispute can be settled by disclosing proofs to the justice. At any stage of the transaction, both parties are either in privelaged situation or can proof to the justice the misbehaviour of the other party. 


\bibliographystyle{alpha}
\bibliography{bibliography}

\end{document}
